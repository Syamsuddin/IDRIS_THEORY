\chapter[Graf Ramanujan--Idris]{Graf Ramanujan--Idris (RJI--N) dan Spektral Informasi}
\label{chap:RJI}

\section{Pendahuluan}
Bab ini memperkenalkan struktur matematis fundamental dari Teori Idris, yaitu 
graf Ramanujan--Idris (RJI--$N$). Graf ini merupakan medium pra-geometrik 
yang menyimpan, mengalirkan, dan memproses informasi pada era Drissian, 
sebelum ruang-waktu dan geometri kontinual muncul sebagai fenomena emergent.

Tujuan bab ini adalah:
\begin{enumerate}
    \item mendefinisikan RJI--$N$ secara formal,
    \item membuktikan kestabilan spektral dan minimalitas derajat $d=3$,
    \item mendefinisikan driston sebagai mode fundamental informasi,
    \item menunjukkan bagaimana geometri emergent lahir dari spektrum $L_I$.
\end{enumerate}

Struktur ini membentuk fondasi bagi aksi Drissian, aliran informasi IRG, 
dan seluruh dinamika emergent yang akan dibahas pada bab-bab selanjutnya.

% ============================================================
\section{Aksioma Struktur Graf}
\label{sec:axioms}
Kita ulangi empat aksioma inti dari Bab I (diringkas untuk konteks graf):

\begin{axiom}[Aksioma Struktur Informasi]
Informasi terdistribusi pada sebuah graf terhingga berderajat tetap.
\end{axiom}

\begin{axiom}[Aksioma Kemetrian Lokal]
Setiap simpul memiliki derajat yang sama: $\deg(v)=d$ untuk semua $v$.
\end{axiom}

\begin{axiom}[Aksioma Homogenitas Spektral]
Operator Laplasian informasi memiliki spektrum stabil terhadap fluktuasi kecil.
\end{axiom}

\begin{axiom}[Aksioma Minimasi Kompleksitas]
Struktur pra-geometrik memiliki kompleksitas minimal yang konsisten dengan 
aksioma di atas.
\end{axiom}

Dari keempat aksioma ini akan diturunkan struktur unik graf RJI--$N$.

% ============================================================
\section{Definisi Formal Graf Ramanujan--Idris}
\label{sec:def_rji}

\begin{definition}[Graf Regular Ramanujan--Idris]
Graf Ramanujan--Idris (RJI--$N$) adalah graf sederhana $G=(V,E)$ dengan
\begin{enumerate}
    \item $|V| = N$,
    \item $\deg(v) = 3$ untuk semua $v$,
    \item adjacency matrix $A$ memenuhi batas Ramanujan:
    \[
        |\lambda_i(A)| \le 2\sqrt{2} \quad \text{untuk semua } i\ge 2.
    \]
\end{enumerate}
\end{definition}

Nilai $d=3$ bukan asumsi ad-hoc; ia muncul dari teorema kestabilan berikut.

% ============================================================
\section{Teorema Kestabilan Spektral Derajat-3}
\label{sec:stability}

\begin{theorem}[Kestabilan Spektral Minimal]
\label{thm:degree3}
Graf regular dengan derajat $d=3$ merupakan nilai integer terkecil dan 
satu-satunya yang secara simultan memenuhi empat kriteria fisika:
\begin{enumerate}
    \item entropi struktural maksimum,
    \item spectral gap optimal,
    \item kestabilan eigenmode fundamental,
    \item non-degenerasi mode dasar,
\end{enumerate}
pada limit $N\to \infty$. Bukti numerik dan analitik tersedia di Bab II,
Lampiran B, serta referensi standar mengenai graf Ramanujan dan spectral graph
theory, misalnya \cite{LPS1988,Hoory2006,Chung1997}.
\end{theorem}

\begin{proof}[Ringkasan Bukti]
Bukti lengkap terdapat pada Lampiran B.  
Secara singkat:
\begin{itemize}
    \item Untuk $d=1,2$ tidak ada spectral gap.
    \item Untuk $d=4$ dan seterusnya, degenerasi mode rendah tidak terhindarkan.
    \item Untuk $d=3$, gap minimum dan entropi maksimum tercapai simultan.
\end{itemize}
\end{proof}

% ============================================================
\section{Operator Laplasian Informasi}
\label{sec:laplacian}

Operator fundamental adalah:
\[
    L_I = d I - \frac{2}{d} A,
\]
yang untuk $d=3$ menjadi
\[
    L_I = 3I - \frac{2}{3}A.
\]

$L_I$ adalah operator Hermitian pada era Drissian kecuali pada fluktuasi
finite-$N$ yang dibahas berikutnya.

% ============================================================
\section{Fluktuasi Non-Hermitian dan Panah Informasi}
\label{sec:nonhermitian}

Pada graf terhingga, fluktuasi arah aliran informasi menyebabkan munculnya 
komponen antisimetri $K_{\text{skew}}$ berskala kecil $O(1/N)$.

Operator efektif:
\[
    L_I^{\text{eff}} = L_I + \varepsilon K_{\text{skew}}, \qquad 
    K_{\text{skew}}^T = -K_{\text{skew}}.
\]
%
\footnote{%
Nilai $\varepsilon$ tidak dipilih secara ad-hoc. Pada Bab V akan diturunkan
bahwa $\varepsilon \sim \ell_P/\ell_H \sim 10^{-61}$, ditentukan secara 
self-consistent oleh dua skala kosmik: panjang Planck dan panjang Hubble.}

\begin{proposition}
Non-Hermitian transien dari $L_I^{\text{eff}}$ menghasilkan panah informasi
(satu-arah) yang, setelah proyeksi ke sub-ruang Hermitian, menjadi identik 
dengan panah waktu termodinamik dan kosmologis.
\end{proposition}

Bukti numerik diberikan pada Bab IV.

% ============================================================
\section{Driston: Mode Fundamental Informasi}
\label{sec:driston}

\begin{definition}[Driston]
Driston adalah eigenmode terkecil nonzero dari operator $L_I$:
\[
    L_I \psi_1 = \lambda_1 \psi_1, \qquad \lambda_1 > 0.
\]
\end{definition}

Pada graf $K_4$ (satu-satunya graf 3-regular dengan $N=4$):
\[
    \text{Spec}(L_I) = \{0,4,4,4\},
\]
sehingga satu driston per $K_4$ unit.

% ============================================================
\section{Metrik dari Spektrum}
\label{sec:metric}

Geometri empat-dimensi emergent diturunkan dari embedding spektral:
\[
    g_{\mu\nu}(x) = \sum_{k=1}^m \lambda_k^{-1} 
    \partial_\mu \psi_k(x)\, \partial_\nu \psi_k(x).
\]

Dimensi spektral dihitung dengan heat trace:
\[
    K(t) = \mathrm{Tr}\, e^{-tL_I} 
    \sim t^{-d_s/2}.
\]

Untuk RJI–$N$:
\[
    d_s \approx 4.
\]

% ============================================================
\section{Operator Markov Informasi}
\label{sec:markov}

Difusi informasi mengikuti:
\[
    \mathcal{M} = I - \varepsilon L_I,
\]
yang memberikan aliran IRG pada Bab V.

% ============================================================
\section{Interpretasi Fisik}
\label{sec:interpret}

\begin{itemize}
    \item Derajat $3$ adalah kondisi minimal stabilitas.
    \item Spektrum $L_I$ menentukan massa, medan, dan dinamika.
    \item Driston adalah kuanta fundamental informasi.
    \item Panah waktu muncul dari fluktuasi non-Hermitian finite-$N$.
    \item Geometri ruang-waktu adalah proyeksi dari spektrum $L_I$.
\end{itemize}

% ============================================================
\section{Kesimpulan Bab III}
\label{sec:conclusion}

Bab ini telah:

\begin{itemize}
    \item Mendefinisikan secara formal Graf Ramanujan--Idris (RJI--$N$) sebagai 
          graf regular berderajat 3 yang memenuhi batas spektral Ramanujan
    \item Membuktikan bahwa derajat 3 adalah nilai unik yang memenuhi kriteria 
          fisika fundamental (entropi maksimal, spectral gap optimal, stabilitas mode)
    \item Mendefinisikan operator Laplasian informasi $L_I = 3I - \frac{2}{3}A$ 
          sebagai operator fundamental yang melahirkan seluruh struktur fisika
    \item Menunjukkan munculnya fluktuasi non-Hermitian transien dan panah informasi
    \item Memperkenalkan driston sebagai mode fundamental informasi ($\lambda_1$)
    \item Menunjukkan bagaimana metrik spektral menghasilkan geometri empat dimensi
\end{itemize}

\textbf{Keterkaitan dengan Bab Berikutnya:}

Bab IV akan membahas Aksi Drissian (SD), yaitu aksi pra-geometrik yang bekerja 
langsung pada graf RJI--$N$. Aksi SD merupakan titik awal dari rantai evolusi 
SD $\to$ WHI $\to$ SP $\to$ SE yang akan dijelaskan secara bertahap. Bab IV 
akan menunjukkan bagaimana dinamika informasional fundamental bekerja pada era 
Drissian, sebelum ruang, waktu, dan geometri muncul.

