%%%%%%%%%%%%%%%%%%%%%%%%%%%%%%%%%%%%%%%%%%%%%%%%%%%%%%%%%%%%
% BAB VI — ERA PLANCKIAN (SP) DAN EMERGENSI AWAL GEOMETRI
%%%%%%%%%%%%%%%%%%%%%%%%%%%%%%%%%%%%%%%%%%%%%%%%%%%%%%%%%%%%

\chapter{Era Planckian (SP) dan Emergensi Awal Geometri}
\label{chap:SP}

Setelah fase transien White Hole Informasi (WHI), sistem memasuki 
\emph{era Planckian} (SP): yaitu fase pertama yang kembali Hermitian 
dan stabil secara spektral. Era SP merupakan jembatan antara dinamika 
informasi pra-geometrik dan era emergen (SE) yang menghasilkan geometri 
kontinu dan, kelak, persamaan Einstein.

%%%%%%%%%%%%%%%%%%%%%%%%%%%%%%%%%%%%%%%%%%%%%%%%%%%%%%%%%%%%
\section{Pemulihan Hermiticity}
%%%%%%%%%%%%%%%%%%%%%%%%%%%%%%%%%%%%%%%%%%%%%%%%%%%%%%%%%%%%

Perturbasi antisimetri $K_{\rm skew}$ yang menjadi ciri era WHI 
meluruh eksponensial:
\begin{equation}
    \varepsilon(t) \longrightarrow 0,
\end{equation}
sehingga operator informasional kembali berbentuk Hermitian murni:
\begin{equation}
    L_I^{(\mathrm{SP})} = 3I - \frac{2}{3}A,
    \label{eq:LISP}
\end{equation}
dengan $A = A^{T}$ \, adjacency matrix dari graf Ramanujan--Idris RJI--$N$ 
(berderajat $3$ dan memenuhi batas Ramanujan).

Spektrum operator $L_I^{(\mathrm{SP})}$ adalah real dan non-negatif:
\begin{equation}
    0 = \lambda_0 < \lambda_1 \le \cdots \le \lambda_{N-1},
\end{equation}
dan mode nol $\psi_0$ mempertahankan interpretasi sebagai arah waktu 
yang telah “dibekukan” selama transisi WHI.

%%%%%%%%%%%%%%%%%%%%%%%%%%%%%%%%%%%%%%%%%%%%%%%%%%%%%%%%%%%%
\section{Dinamika Mode Informasi}
%%%%%%%%%%%%%%%%%%%%%%%%%%%%%%%%%%%%%%%%%%%%%%%%%%%%%%%%%%%%

Setelah pemulihan Hermiticity, dinamika efektif mode informasional 
dikendalikan oleh persamaan evolusi linear:
\begin{equation}
    \dot{\psi}_k(t) = - \lambda_k \, \psi_k(t),
    \qquad k \ge 1.
    \label{eq:psiSP}
\end{equation}

Solusi umum:
\begin{equation}
    \psi_k(t) = \psi_k(0)\, e^{-\lambda_k t}.
\end{equation}

Karakteristiknya:
\begin{itemize}
    \item Mode rendah ($\lambda_k$ kecil) meluruh lambat dan menentukan struktur 
    geometri emergen (era SE).
    \item Mode tinggi ($\lambda_k$ besar) meluruh cepat dan menyumbang fluktuasi 
    energi awal (analisis penuh pada Bab XVII–XX).
    \item Mode nol $\psi_0$ bersifat konstan dan berperan sebagai arah waktu 
    makroskopik.
\end{itemize}

%%%%%%%%%%%%%%%%%%%%%%%%%%%%%%%%%%%%%%%%%%%%%%%%%%%%%%%%%%%%
\section{Metrik Spektral Awal}
%%%%%%%%%%%%%%%%%%%%%%%%%%%%%%%%%%%%%%%%%%%%%%%%%%%%%%%%%%%%

Pada era SP, struktur geometri mulai muncul melalui embedding spektral 
mode non-nol:
\begin{equation}
    g_{\mu\nu}^{\mathrm{(SP)}}(x) =
    \sum_{k\ge 1}
    \lambda_k^{-1}
    \left(
    \partial_\mu \psi_k(x)
    \right)
    \left(
    \partial_\nu \psi_k(x)
    \right).
    \label{eq:gSP}
\end{equation}

Sifat-sifat:
\begin{enumerate}
    \item \textbf{Tanda Lorentzian}:  
    arah waktu telah ditentukan saat WHI melalui mode nol imajiner 
    $\lambda_0 \to i\varepsilon$, sehingga tanda metrik efektif adalah $(-,+,+,+)$.
    \item \textbf{Dimensi efektif empat}:  
    dimensi spektral RJI--$N$ adalah $d_s \approx 4$ 
    (Bab II), sehingga limit kontinuum menghasilkan ruang-waktu 
    empat dimensi.
    \item \textbf{Diskrit $\to$ kontinu}:  
    pada limit $N\to\infty$, metrik spektral~(\ref{eq:gSP}) 
    menjadi metrik smooth pada manifold kontinu empat dimensi.
\end{enumerate}

%%%%%%%%%%%%%%%%%%%%%%%%%%%%%%%%%%%%%%%%%%%%%%%%%%%%%%%%%%%%
\section{Transisi SP ke Era Emergen (SE)}
%%%%%%%%%%%%%%%%%%%%%%%%%%%%%%%%%%%%%%%%%%%%%%%%%%%%%%%%%%%%

Transisi dari era SP menuju era emergent (SE) terjadi ketika:
\begin{enumerate}
    \item $N$ cukup besar sehingga limit kontinuum dapat diambil;
    \item mode rendah mendominasi dinamika global;
    \item kontribusi mode tinggi menjadi energi fluktuatif teredam;
    \item metrik spektral~(\ref{eq:gSP}) menjadi smooth dan 
          kompatibel dengan struktur manifold.
\end{enumerate}

Pada kondisi ini, geometri spektral RJI--$N$ memasuki 
regime di mana persamaan Einstein muncul sebagai batas kontinum.  
Bukti lengkap derivasi Einstein–Idris disajikan pada 
Bab X–XI dan Lampiran D.

%%%%%%%%%%%%%%%%%%%%%%%%%%%%%%%%%%%%%%%%%%%%%%%%%%%%%%%%%%%%
\section{Kesimpulan}
%%%%%%%%%%%%%%%%%%%%%%%%%%%%%%%%%%%%%%%%%%%%%%%%%%%%%%%%%%%%

Era Planckian (SP) merupakan fase awal stabil pasca-transisi WHI dan 
menjadi fondasi geometri emergen. Ciri utamanya:
\begin{itemize}
    \item operator kembali Hermitian,
    \item dinamika mode informasional eksponensial,
    \item pembentukan awal metrik spektral Lorentzian,
    \item persiapan transisi menuju geometri kontinuum (SE),
    \item tanpa membuat klaim kosmologis prematur (energi gelap, dark matter, atau 
          persamaan Einstein eksplisit).
\end{itemize}

Bab berikutnya menguraikan secara rinci era emergen (SE), termasuk 
bagaimana geometri kontinu sepenuhnya muncul dari struktur informasional 
RJI--$N$ dan bagaimana limit kontinuum menghasilkan persamaan Einstein 
sebagai teori efektif makroskopik.