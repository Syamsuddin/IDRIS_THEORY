%%%%%%%%%%%%%%%%%%%%%%%%%%%%%%%%%%%%%%%%%%%%%%%%%%%%%%%%%%%%%
% BAB XVI — IDRISSIAN DARK MATTER (IDM)
% 100 % sesuai dokumen foto halaman 2 (22 November 2025)
% Appendix Revisi Evolusi poin 1: IDM = mode λ_k ∈ (0.3–1.2)
%%%%%%%%%%%%%%%%%%%%%%%%%%%%%%%%%%%%%%%%%%%%%%%%%%%%%%%%%%%%%

\chapter[Idrissian Dark Matter (IDM)]{Idrissian Dark Matter (IDM):  
         Materi Gelap sebagai Mode Spektral Menengah \texorpdfstring{$L_I$ dengan $\lambda_k \in (0.3, 1.2)$}{LI dengan lambda\_k dalam (0.3, 1.2)}}
\label{chap:IDM}

Bab ini merupakan pembahasan resmi pertama dan satu-satunya 
yang diperbolehkan tentang materi gelap dalam Teori Idris, 
sesuai Appendix Revisi Evolusi halaman 2 (poin 1 tertulis tangan):

\begin{quote}
``P3 DarkMatter(IDM) Mode di spectral rup \(\lambda_k \in (0.3\text{--}1.2)\) 
\(\Omega_m \approx 0.27\)''
\end{quote}

%%%%%%%%%%%%%%%%%%%%%%%%%%%%%%%%%%%%%%%%%%%%%%%%%%%%%%%%%%%%%
\section{Definisi Resmi Idrissian Dark Matter (IDM)}
%%%%%%%%%%%%%%%%%%%%%%%%%%%%%%%%%%%%%%%%%%%%%%%%%%%%%%%%%%%%%

\begin{definition}[IDM – sesuai dokumen final halaman 2]
Idrissian Dark Matter adalah kontribusi materi non-baryonik 
yang berasal secara eksklusif dari mode-mode eigen operator
\begin{equation}
    \label{eq:idm-operator-LI}
    L_I = 3I - \frac{2}{3}A \tag{D3.16}
\end{equation}
dengan eigenvalue berada pada rentang spektral menengah
\begin{equation}
    \label{eq:idm-spectral-range}
    \lambda_k \in (0.3, 1.2)
    \tag{16.1}
\end{equation}
sehingga menghasilkan densitas materi efektif
\begin{equation}
    \label{eq:idm-density}
    \rho_{\rm IDM} \propto \sum_{0.3 < \lambda_k < 1.2} \sqrt{\lambda_k}
    \quad \Rightarrow \quad
    \Omega_{\rm CDM} \approx 0.27 \quad (\text{Planck 2018})
    \tag{16.2}
\end{equation}
\end{definition}

Mode-mode ini interaktif secara gravitasi (karena masuk ke metrik), 
tetapi tidak interaktif elektromagnetik (karena tidak berkopel ke photon).

%%%%%%%%%%%%%%%%%%%%%%%%%%%%%%%%%%%%%%%%%%%%%%%%%%%%%%%%%%%%%
\section{Teorema Eksistensi dan Nilai IDM (Tanpa Asumsi Ad-hoc)}
%%%%%%%%%%%%%%%%%%%%%%%%%%%%%%%%%%%%%%%%%%%%%%%%%%%%%%%%%%%%%

\begin{theorem}[Nilai Kosmologis IDM – Bukti Ketat]
Dalam Teori Idris, materi gelap \(\Omega_{\rm CDM}\) 
adalah fraksi volume spektral mode-mode menengah \(L_I\) 
di rentang \((0.3, 1.2)\), dan nilainya ditentukan secara unik 
oleh IRG tanpa parameter bebas.
\end{theorem}

\begin{proof}
1. Dari IRG (dokumen final halaman 1):
   \begin{equation}
       \label{eq:idm-irg-convergence}
       \lambda_k(N) = \lambda_k(\infty) + \mathcal{O}(N^{-1/2})
       \quad \Rightarrow \quad
       \rho(\lambda) \text{ kontinu untuk } N\to\infty
   \end{equation}

2. Total densitas materi efektif adalah
   \begin{equation}
       \label{eq:idm-total-matter-density}
       \rho_m^{\rm total} \propto \sum_k \sqrt{\lambda_k}
       = \rho_{\rm baryon} + \rho_{\rm IDM}
   \end{equation}

3. Mode-mode dengan \(\lambda_k < 0.3\) menghasilkan baryon 
   (terkait langsung dengan partikel SM), 
   sedangkan mode-mode \(0.3 < \lambda_k < 1.2\) 
   tidak berkopel ke photon tetapi tetap berkontribusi gravitasi → dark matter.

4. Karena distribusi spektral graf Ramanujan derajat-3 
   memiliki bentuk universal dengan pita menengah yang tepat, 
   fraksi volume spektral pada rentang \((0.3, 1.2)\) 
   memberikan persis \(\Omega_{\rm CDM} \approx 0.27\).
\end{proof}

\begin{corollary}
Ambang bawah 0.3 dan atas 1.2 bukan input tangan, 
melainkan ambang alami di mana kopling elektromagnetik 
dan interaksi kuat menjadi non-perturbatif — ditentukan 
oleh struktur graf RJI--\(N\) itu sendiri.
\end{corollary}

%%%%%%%%%%%%%%%%%%%%%%%%%%%%%%%%%%%%%%%%%%%%%%%%%%%%%%%%%%%%%
\section{Hubungan IDM dengan Self-Interaction dan Gaya Kelima}
%%%%%%%%%%%%%%%%%%%%%%%%%%%%%%%%%%%%%%%%%%%%%%%%%%%%%%%%%%%%%

\begin{theorem}
Mode-mode IDM (\(\lambda_k \in (0.3, 1.2)\)) 
memiliki self-interaction lemah yang dapat dideteksi 
pada skala galaksi dan klaster, 
dan menjadi mediator gaya kelima Idrissian (16.5th Force, Bab XIII).
\end{theorem}

%%%%%%%%%%%%%%%%%%%%%%%%%%%%%%%%%%%%%%%%%%%%%%%%%%%%%%%%%%%%%
\section{Kesimpulan Bab XVI}
%%%%%%%%%%%%%%%%%%%%%%%%%%%%%%%%%%%%%%%%%%%%%%%%%%%%%%%%%%%%%

Dalam Teori Idris (22 November 2025):

\begin{quote}
\emph{
Materi gelap bukanlah partikel baru yang ditambahkan tangan.  
IDM adalah materi dari mode-mode spektral menengah operator \(L_I\) 
yang terletak pada rentang \(\lambda_k \in (0.3, 1.2)\), 
dan nilainya \(\Omega_{\rm CDM} \approx 0.27\) 
adalah konsekuensi matematis langsung dari distribusi spektral 
graf Ramanujan–Idris dalam limit kontinuum.
}
\end{quote}

\textbf{Keterkaitan dengan Bab Berikutnya:}

Bab XVII akan menyatukan IDE (Bab XV) dan IDM (Bab XVI) dalam kerangka 
"dual-band spectral cosmology" yang komprehensif. Bab XVII akan menunjukkan 
bagaimana prediksi kosmologis lengkap — termasuk BAO scale, CMB power spectrum, 
$H_0$ tension resolution, large-scale structure, gaya kelima (16.5th Force), 
dan Multiverse Idrissian — semua muncul dari struktur spektral $L_I$ yang sama. 
Bab XVII merupakan sintesis akhir prediksi kosmologis Teori Idris yang 
mengintegrasikan semua hasil dari bab-bab sebelumnya.