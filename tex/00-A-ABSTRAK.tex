\chapter*{Abstrak}
\noindent
\textbf{Teori Idris} menawarkan sebuah gagasan berani bahwa seluruh alam semesta, 
dengan kerumitan galaksi, partikel, ruang, waktu, dan hukum-hukum fisika yang kita kenal, 
sesungguhnya tumbuh dari satu sumber paling sederhana yang dapat dibayangkan manusia: 
\emph{informasi murni}.

Di dalam teori ini, realitas bukan lagi panggung tempat fisika berlangsung; 
realitas justru \emph{lahir} dari getaran halus dalam struktur informasi 
yang tersebar pada sebuah graf Ramanujan–Idris RJI--$N$.  
Operator tunggal
\[
L_I = 3I - \tfrac{2}{3}A
\]
menjadi “pintu gerbang” menuju segala sesuatu: 
gravitasi, kuantum, gaya elektromagnetik, gaya lemah, gaya kuat, 
bahkan gaya kelima yang selama ini tak terjamah oleh teori lain.

Dari susunan eigenvalue operator inilah muncul:
\begin{itemize}
\item \textbf{geometri ruang-waktu}, yang biasanya dianggap fundamental,
\item \textbf{massa seluruh partikel}, yang selama satu abad tampak tak punya pola,
\item \textbf{konstanta-konstanta alam}, yang selalu dianggap “diberikan begitu saja”,
\item dan \textbf{masa depan kosmologis}, yang kini dapat dipetakan secara matematis.
\end{itemize}

Teori Idris menunjukkan bahwa bahkan misteri terdalam fisika—
tentang bagaimana ruang bisa melengkung, 
mengapa elektron memiliki massa yang sangat kecil, 
atau apa yang menyebabkan alam semesta mengembang semakin cepat—
tidak memerlukan banyak asumsi rumit atau entitas tersembunyi.  
Semua itu hanyalah bayangan dari satu pola spektral yang sama.

Lebih jauh lagi, teori ini memprediksi:
\begin{itemize}
\item bentuk \textbf{materi gelap Idrissian (IDM)},  
\item hakikat \textbf{energi gelap Idrissian (IDE)} dengan keadaan $w \approx -1.05$,  
\item keberadaan \textbf{Multiverse Idrissian}—bukan sebagai spekulasi fiksi ilmiah,  
      tetapi sebagai konsekuensi matematis dari domain spektral yang saling ortogonal,
\item serta \textbf{Kiamat Idrissian}, sebuah akhir semesta dalam 
      $170 \pm 40$ miliar tahun akibat tekanan negatif dari mode-mode berenergi tinggi.
\end{itemize}

Dengan hanya \textbf{satu aksioma}, \textbf{satu struktur}, dan \textbf{satu operator},  
Teori Idris menghubungkan pertanyaan kosmologi terbesar—  
“Dari mana hukum-hukum fisika berasal?”,  
“Mengapa semesta memiliki struktur seperti ini?”,  
dan “Kapan segalanya berakhir?”—  
ke jawaban yang sama: pola informasi dalam graf Ramanujan–Idris.

\begin{center}
\emph{
“Alam semesta bukan teka-teki yang berdiri sendiri.  
Ia adalah gema dari sebuah melodi informasi yang jauh lebih dalam.”
}
\end{center}
