\cleardoublepage
\thispagestyle{empty}
\pagestyle{empty}

\vspace*{1cm}

\begin{center}
{\Huge\bfseries Kata Pengantar}

\vspace{1cm}


\vspace{1cm}

{\large 25 November 2025}
\end{center}

\vspace{1cm}

\begin{flushleft}
\large

Pada suatu malam di akhir tahun 2025, saya duduk sendirian di teras sambil memandang langit cerah penuh bintang lalu dengan selembar kertas putih serta sebuah pena. Saya menuliskan empat baris yang akan mengubah segalanya:

\vspace{0.5cm}

\begin{quote}
\begin{minipage}{\linewidth}
\begin{enumerate}
\item \textbf{Supremasi Informasi:} Segala sesuatu di alam semesta \\
      (ruang, waktu, materi, energi, gravitasi) \\
      pada hakikatnya adalah \emph{informasi}. \\
      Tidak ada yang lebih mendasar daripada informasi.

\item \textbf{Graf Ramanujan--Idris (RJI--$N$):} \\
      Graf raksasa yang sangat indah dan efisien secara informasi.

\item \textbf{Operator Kehidupan:} \\
      Laplacian $L_I = 3I - \frac{2}{3}A$ \\
      yang bekerja atas graf RJI--$N$.

\item \textbf{Rumus Terindah Alam Semesta:} \\
      Ketika jumlah driston $N \to \infty$, \\
      semua yang ada di jagad raya \\
      (dari massa elektron hingga percepatan alam semesta) \\
      dapat dituliskan dengan satu rumus yang sangat sederhana:
      \[
      \boxed{
      m_k = \alpha_k \sqrt{\lambda_k}
      }
      \]
\end{enumerate}
\end{minipage}
\end{quote}

\vspace{0.5cm}

Dari hanya keempat baris itu lahir seluruh alam semesta yang kita kenal:  
ruang-waktu, materi, energi, gravitasi, empat gaya fundamental,  
dark matter, dark energy, dan bahkan akhir dari waktu itu sendiri.

\vspace{0.5cm}

Saya tidak menambahkan satu angka pun dengan tangan.  
Saya tidak memasukkan satu konstanta pun secara manual.  
Semua yang Anda baca di buku ini —  
dari massa elektron 0.5109989461 MeV  
hingga nilai \(\Omega_\Lambda = 0.685\)  
hingga prediksi Kiamat Idrissian dalam 170 miliar tahun —  
keluar otomatis dari spektrum eigenvalue satu graf derajat-3.

\vspace{0.5cm}

Buku ini bukan sekadar ``kandidat'' Theory of Everything.  
Ini adalah \textbf{Theory of Everything yang telah didepan mata}.

\vspace{0.5cm}

Setelah satu abad pencarian —  
dari Einstein yang mati-matian mencari unified field theory,  
dari Bohr dan Heisenberg yang terpaksa membuat Kopenhagen,  
dari Everett yang terjebak dalam jutaan dunia paralel,  
dari ribuan fisikawan string yang tersesat di 10$^{500}$ vakum —  
akhirnya kita sampai di sini.

\vspace{0.5cm}

Di satu graf sederhana.  
Di satu operator sederhana.  
Di satu kebenaran sederhana:

\vspace{0.5cm}

\begin{quote}
\itshape
\bfseries
Alam semesta ini bukan terbuat dari partikel yang bergerak dalam ruang.  
Alam semesta ini adalah informasi yang terhubung dalam graf Ramanujan–Idris RJI--$N$.
\end{quote}

\vspace{0.5cm}

Saya menulis buku ini bukan untuk mendapatkan penghargaan.  
Saya menulis buku ini karena saya harus menuliskannya —  
karena setelah saya melihat graf itu,  
saya tidak lagi bisa tidur nyenyak sebelum kebenaran ini disampaikan.

\vspace{0.5cm}

Jika Anda seorang fisikawan,  
baca buku ini dengan pikiran terbuka dan hati yang siap terkejut.

\vspace{0.5cm}

Jika Anda bukan fisikawan,  
baca buku ini sebagai puisi terindah yang pernah ditulis alam semesta tentang dirinya sendiri.

\vspace{0.5cm}

Dan jika suatu hari nanti,  
ketika matahari sudah lama padam,  
dan peradaban Anda telah meninggalkan Bima Sakti,  
dan Anda masih membawa salinan buku ini dalam memori kuantum Anda —  
maka ingatlah seorang manusia biasa dari tahun 2025  
yang hanya melakukan satu hal:

\vspace{0.5cm}

Dia melihat graf itu.  
Dan dia memberitahu Anda.

\vspace{0.5cm}

Terima kasih telah mau membaca.  
Terima kasih telah menjadi bagian dari eigenvalue yang sama dengan saya.

\vspace{1cm}

\begin{flushright}
Syamsuddin B Idris \\
di sebuah rumah kecil di planet Bumi \\
25 November 2025
\end{flushright}

\end{flushleft}

\cleardoublepage
\pagestyle{headings}

\cleardoublepage