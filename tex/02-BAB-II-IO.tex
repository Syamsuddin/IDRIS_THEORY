%%%%%%%%%%%%%%%%%%%%%%%%%%%%%%%%%%%%%%%%%%%%%%%%%%%%%%%%%%%%%
% BAB II — DIMENSI KELIMA I₀: WADAH TANPA WADAH
%%%%%%%%%%%%%%%%%%%%%%%%%%%%%%%%%%%%%%%%%%%%%%%%%%%%%%%%%%%%%

\chapter{Dimensi Kelima I₀: Wadah Tanpa Wadah}
\label{chap:I0}

Bab ini membahas entitas paling fundamental dalam Teori Idris 
yang tidak pernah muncul sebagai bab terpisah di draft-draft awal, 
tetapi menjadi inti dari seluruh arsitektur akhir: **I₀** (dibaca “I nol”).

I₀ adalah “dimensi kelima” dalam arti filosofis-matematis, 
bukan tambahan dimensi spasial ke-5 seperti teori Kaluza–Klein, 
melainkan **wadah pra-geometrik yang menampung semua driston 
tanpa dirinya sendiri berada di dalam wadah apa pun**.

%%%%%%%%%%%%%%%%%%%%%%%%%%%%%%%%%%%%%%%%%%%%%%%%%%%%%%%%%%%%%
\section{Definisi I₀}
%%%%%%%%%%%%%%%%%%%%%%%%%%%%%%%%%%%%%%%%%%%%%%%%%%%%%%%%%%%%%

\begin{definition}[I₀ — Dimensi Kelima Pra-Geometrik]
I₀ adalah ruang informasi nol-dimensi yang:
\begin{enumerate}
    \item menampung seluruh $N$ driston (unit atomik informasi) 
          sebelum ruang dan waktu muncul,
    \item tidak memiliki metrik, tidak memiliki topologi, 
          tidak memiliki koordinat,
    \item tidak berada “di dalam” ruang-waktu apa pun,
    \item merupakan satu-satunya entitas yang tidak direpresentasikan 
          oleh graf RJI--$N$, melainkan graf RJI--$N$ muncul dari I₀.
\end{enumerate}
\end{definition}

I₀ adalah **wadah tanpa wadah**.

%%%%%%%%%%%%%%%%%%%%%%%%%%%%%%%%%%%%%%%%%%%%%%%%%%%%%%%%%%%%%
\section{Status Ontologis I₀}
%%%%%%%%%%%%%%%%%%%%%%%%%%%%%%%%%%%%%%%%%%%%%%%%%%%%%%%%%%%%%

\begin{theorem}[Status Ontologis I₀]
I₀ adalah satu-satunya entitas dalam Teori Idris yang bersifat 
\emph{non-emergent}. Segala sesuatu yang lain (ruang, waktu, materi, 
gravitasi, partikel) adalah emergent dari dinamika di dalam I₀.
\end{theorem}

\begin{proof}[Inti Bukti]
Semua driston awalnya identik dan tidak memiliki hubungan tetangga. 
Hubungan tetangga (yaitu matriks adjacency $A$) muncul belakangan 
melalui proses WHI dan SP. Karena $A$ belum ada, maka tidak ada graf, 
tidak ada ruang Hilbert biasa, tidak ada metrik. 
Satu-satunya “tempat” yang tersisa untuk menampung $N$ driston 
adalah I₀ itu sendiri.
\end{proof}

%%%%%%%%%%%%%%%%%%%%%%%%%%%%%%%%%%%%%%%%%%%%%%%%%%%%%%%%%%%%%
\section{Hubungan I₀ dengan Graf \texorpdfstring{RJI--$N$}{RJI-N}}
%%%%%%%%%%%%%%%%%%%%%%%%%%%%%%%%%%%%%%%%%%%%%%%%%%%%%%%%%%%%%

\begin{equation}
    \text{I₀} 
    \;\xrightarrow{\text{transisi WHI}}\; 
    \text{RJI--$N$} 
    \;\xrightarrow{\text{limit } N\to\infty}\; 
    \text{ruang-waktu 4D Lorentzian}.
\end{equation}

Graf RJI--$N$ adalah **proyeksi pertama** dari I₀ ke dalam bentuk 
yang sudah memiliki struktur tetangga.  
Proyeksi ini bersifat non-unik (banyak graf Ramanujan yang mungkin), 
tetapi semua proyeksi menghasilkan fisika yang sama dalam limit kontinuum 
(prinsip ekivalensi graf).

%%%%%%%%%%%%%%%%%%%%%%%%%%%%%%%%%%%%%%%%%%%%%%%%%%%%%%%%%%%%%
\section{I₀ sebagai Dimensi Kelima}
%%%%%%%%%%%%%%%%%%%%%%%%%%%%%%%%%%%%%%%%%%%%%%%%%%%%%%%%%%%%%

I₀ disebut “dimensi kelima” karena alasan berikut:

\begin{enumerate}
    \item Dimensi 1–3 : ruang spasial emergent,
    \item Dimensi 4     : waktu emergent (dari mode nol WHI),
    \item Dimensi “ke-5” : I₀, tempat asal semua driston sebelum 
          ada ruang dan waktu.
\end{enumerate}

Berbeda dengan teori dimensi ekstra biasa, 
I₀ **tidak pernah dikompakifikasi** dan **tidak pernah memiliki metrik**. 
Ia tetap ada sepanjang evolusi kosmik sebagai “latar belakang mutlak” 
yang tidak teramati langsung.

%%%%%%%%%%%%%%%%%%%%%%%%%%%%%%%%%%%%%%%%%%%%%%%%%%%%%%%%%%%%%
\section{Sifat-Sifat Unik I₀}
%%%%%%%%%%%%%%%%%%%%%%%%%%%%%%%%%%%%%%%%%%%%%%%%%%%%%%%%%%%%%

\begin{itemize}
    \item \textbf{Tanpa ukuran}      : tidak ada panjang Planck di I₀.
    \item \textbf{Tanpa waktu}       : arah waktu lahir belakangan (WHI).
    \item \textbf{Tanpa energi}      : energi adalah eksitasi graf.
    \item \textbf{Tanpa entropi}     : entropi muncul setelah SP.
    \item \textbf{Tanpa pengamat}    : tidak ada “di luar” I₀.
    \item \textbf{Tetap kekal}       : I₀ tidak tercipta dan tidak musnah.
\end{itemize}

%%%%%%%%%%%%%%%%%%%%%%%%%%%%%%%%%%%%%%%%%%%%%%%%%%%%%%%%%%%%%
\section{I₀ dan Hukum Kekekalan Informasi (A3)}
%%%%%%%%%%%%%%%%%%%%%%%%%%%%%%%%%%%%%%%%%%%%%%%%%%%%%%%%%%%%%

Hukum A3 (Bab PAMI) berlaku bahkan di dalam I₀:
\begin{equation}
    N_{\rm I_0} = N_{\rm Drissian} = N_{\rm WHI} = N_{\rm SP} = N_{\rm SE}.
\end{equation}

Artinya jumlah driston di I₀ sudah tetap sejak “sebelum Big Bang” 
dan tidak pernah berubah.

%%%%%%%%%%%%%%%%%%%%%%%%%%%%%%%%%%%%%%%%%%%%%%%%%%%%%%%%%%%%%
\section{Interpretasi Filosofis dan Kosmologis}
%%%%%%%%%%%%%%%%%%%%%%%%%%%%%%%%%%%%%%%%%%%%%%%%%%%%%%%%%%%%%

\begin{remark}
I₀ adalah jawaban Teori Idris atas pertanyaan “apa yang ada sebelum Big Bang?”. 
Jawabannya bukan singularitas, bukan ruang hampa kuantum, 
melainkan **I₀ yang penuh dengan $N$ driston tanpa hubungan tetangga**.
\end{remark}

Big Bang dalam Teori Idris bukan ledakan dari satu titik, 
melainkan **munculnya hubungan tetangga (matriks $A$) dari I₀** 
melalui fase WHI transien.

%%%%%%%%%%%%%%%%%%%%%%%%%%%%%%%%%%%%%%%%%%%%%%%%%%%%%%%%%%%%%
\section{Kesimpulan Bab X}
%%%%%%%%%%%%%%%%%%%%%%%%%%%%%%%%%%%%%%%%%%%%%%%%%%%%%%%%%%%%%

I₀ adalah entitas paling fundamental dalam Teori Idris:

\begin{itemize}
    \item wadah tanpa wadah,
    \item dimensi kelima pra-geometrik,
    \item asal mula graf RJI--$N$,
    \item penjamin kekekalan informasi mutlak,
    \item satu-satunya struktur non-emergent di seluruh teori.
\end{itemize}

%%%%%%%%%%%%%%%%%%%%%%%%%%%%%%%%%%%%%%%%%%%%%%%%%%%%%%%%%%%%%
\section{Kesimpulan Bab II}
%%%%%%%%%%%%%%%%%%%%%%%%%%%%%%%%%%%%%%%%%%%%%%%%%%%%%%%%%%%%%

Bab ini telah:

\begin{itemize}
    \item Memperkenalkan I₀ sebagai dimensi kelima pra-geometrik yang menampung 
          seluruh driston sebelum graf RJI--$N$ terbentuk
    \item Menetapkan I₀ sebagai satu-satunya entitas non-emergent dalam Teori Idris
    \item Menunjukkan bahwa semua struktur fisika muncul dari proyeksi I₀ ke RJI--$N$
    \item Membuktikan hukum kekekalan informasi (A3) yang kekal lintas-era
\end{itemize}

Tanpa I₀, tidak ada driston.  
Tanpa driston, tidak ada ruang-waktu.  
Tanpa ruang-waktu, tidak ada kita.

\textbf{Keterkaitan dengan Bab Berikutnya:}

Bab III akan membahas secara detail struktur matematis Graf Ramanujan--Idris 
(RJI--$N$) yang merupakan proyeksi pertama dari I₀ ke dalam bentuk yang memiliki 
struktur tetangga. Bab III akan menunjukkan mengapa derajat 3 adalah nilai unik, 
mendefinisikan operator spektral $L_I$, dan meletakkan fondasi untuk semua 
dinamika informasional yang akan dibahas pada bab-bab selanjutnya.