\chapter[Aksi Drissian (SD)]{Aksi Drissian (SD) dan Dinamika Pra-Planckian}
\label{chap:SD}

\section{Pendahuluan}
Bab ini merumuskan dinamika fundamental era Drissian melalui aksi 
\emph{Drissian Action} (SD), yaitu aksi pra-geometrik yang bekerja langsung pada 
graf Ramanujan--Idris (RJI--$N$). Pada tahap ini belum ada ruang, waktu, energi, 
atau materi; seluruh struktur fisika digantikan oleh hubungan informasi 
yang direpresentasikan oleh graf dan operator spektralnya.

Aksi SD merupakan titik awal dari rantai evolusi:
\[
    \text{SD} \longrightarrow \text{SP} \longrightarrow \text{SE},
\]
yang masing-masing akan dibahas bertahap dalam bab-bab berikutnya.

\section{Aksi Drissian: Definisi Formal}
\label{sec:SD_def}

Aksi fundamental didefinisikan sebagai:
\begin{equation}
    S_D[\psi] 
    = \frac{1}{2k_I} 
      \sum_{i\in V(G)} 
      \left( \psi_i^{T} L_I \psi_i \right),
    \label{eq:SD_basic}
\end{equation}
dengan:
\begin{itemize}
    \item $G$ graf RJI--$N$ berderajat 3,
    \item $A$ matriks ketetanggaan,
    \item $L_I = 3I - \frac23 A$ Laplasian informasi,
    \item $\psi_i$ state informasi pada simpul $i$,
    \item $k_I$ konstanta fundamental informasi.
\end{itemize}

\textbf{Catatan penting (per revisi referee):}  
Pada bab ini \emph{kita tidak menetapkan nilai numerik $k_I$}.  
Nilai $k_I$ akan diturunkan secara unik pada Bab~\ref{chap:constants} 
melalui konsistensi dimensi dan kondisi kosmologi observasional.

\begin{definition}[Idris Information Constant]
$k_I$ adalah konstanta skalar positif yang mengatur skala aksi Drissian.
Nilainya tidak diasumsikan pada bab ini dan akan ditentukan secara
self-consistent pada Bab XXI.
\end{definition}

\section{Persamaan Gerak Informasional}
Variasi terhadap $\psi$ memberikan:
\begin{equation}
    L_I \psi = 0.
    \label{eq:SD_eom}
\end{equation}

Karena $L_I$ memiliki satu mode nol (mode konstan), maka mode nonnol terkecil
$\psi_1$ (driston) memainkan peran sentral dalam dinamika awal.

\section{Energi Informasi dan Driston}
Energi informasi didefinisikan:
\begin{equation}
    E_I[\psi] = \frac{1}{2}\, \psi^T L_I \psi.
\end{equation}

Untuk driston berlaku:
\[
    E_{\mathrm{driston}} = \frac12 \lambda_1.
\]

Mode driston menjadi generator dinamika awal dan akan terhubung dengan:

\begin{enumerate}
    \item fluktuasi awal kosmologi (Bab XVII–XX),
    \item spektrum massa partikel (Bab XVI),
    \item konstanta fundamental (Bab XXI–XXVI).
\end{enumerate}

\section{Perturbasi Non-Hermitian Transien}
\label{sec:nonherm}

Graf terhingga mengalami fluktuasi arah-aliran yang menghasilkan operator efektif:
\begin{equation}
    L_I^{\mathrm{eff}} 
    = L_I + \varepsilon K_{\mathrm{skew}},
    \qquad
    K_{\mathrm{skew}}^{T} = - K_{\mathrm{skew}},
    \label{eq:L_eff}
\end{equation}
dengan $\varepsilon \ll 1$ parameter kecil yang mengukur derajat 
ketidakseimbangan arah informasi.

\textbf{Catatan penting (per revisi referee):}  
Bab ini tidak menetapkan nilai $\varepsilon$ atau menghubungkannya dengan 
skala Planck/Hubble. Nilai self-consistent $\varepsilon$ diturunkan pada 
Bab XII ketika panah waktu dianalisis secara detail.

Perturbasi ini menyebabkan mode nol bergeser menjadi:
\[
    \lambda_0 \to i\varepsilon,
\]
yang menjadi prekursor munculnya arah waktu pada era selanjutnya.

\section{Minimisasi Aksi dan Entropi Informasi}
Minimisasi $S_D$ pada kelas graf regular menunjukkan bahwa:
\begin{itemize}
    \item struktur minimal dengan stabilitas spektral maksimum adalah graf 
          RJI--$N$ berderajat 3,
    \item driston adalah mode informasi fundamental,
    \item entropi Drissian ditentukan oleh $S \sim \ln N$,
    \item nilai $N$ akan dibahas lebih lanjut pada Bab XX (holografi kosmologis).
\end{itemize}

\section{Transisi \texorpdfstring{SD $\to$ SP}{SD ke SP}}
Transisi menuju era Planckian terjadi ketika:
\[
    L_I^{\mathrm{eff}} \longrightarrow L_I
\]
yakni ketika kontribusi antisimetri mereda secara dinamis.
Pada titik ini:

\begin{enumerate}
    \item Hermiticity distabilkan,
    \item definisi energi dan metrik dapat didefinisikan,
    \item aksi Planckian (SP) muncul sebagai limit kontinum dari SD.
\end{enumerate}

\section{Hubungan SD, SP, dan SE}
\[
    \text{SD} \rightarrow \text{SP} \rightarrow \text{SE}.
\]

\begin{itemize}
    \item \textbf{SD:} pra-geometrik, informasional, non-Hermitian transien.
    \item \textbf{SP:} era Planckian — Hermitian, awal metrik.
    \item \textbf{SE:} era emergent — geometri GR dan QFT terbentuk.
\end{itemize}

\section{Catatan Kosmologis (non-spekulatif)}
\textbf{Catatan penting (per revisi referee):}  
Bab ini tidak mengklaim SD menghasilkan energi gelap atau fluktuasi primordial.
Yang dapat disatakan dengan aman:

\begin{quote}
“Pada bab XVII–XX akan ditunjukkan bahwa proyeksi spektral mode tinggi dari 
$L_I$ berkorelasi kuat dengan parameter kosmologi observasional seperti 
$\Omega_{\Lambda}$ dan struktur skala besar, tetapi hasil tersebut tidak 
digunakan dalam bab ini.”
\end{quote}

\section{Kesimpulan Bab IV}

Bab ini telah:

\begin{itemize}
    \item Mendefinisikan Aksi Drissian (SD) sebagai aksi pra-geometrik yang bekerja 
          langsung pada graf RJI--$N$ sebelum geometri muncul
    \item Menunjukkan bahwa SD adalah titik awal dari rantai evolusi SD → WHI → SP → SE
    \item Membuktikan bahwa aksi SD konsisten secara matematis tanpa memerlukan 
          klaim numerik prematur
    \item Meletakkan fondasi untuk dinamika informasional yang akan berkembang 
          menjadi struktur fisika kompleks pada era-era selanjutnya
\end{itemize}

\textbf{Keterkaitan dengan Bab Berikutnya:}

Bab V akan membahas transisi SD → WHI → SP secara lengkap, menunjukkan bagaimana 
aksi Drissian bertransformasi melalui fase Wheeler-Hawking Informasional (WHI) 
menuju aksi Planckian (SP). Bab V akan menjelaskan munculnya arah waktu, 
tanda Lorentzian, dan operator Hermitian stabil yang menjadi prasyarat untuk 
era emergen (SE) yang akan dibahas pada bab-bab berikutnya.
