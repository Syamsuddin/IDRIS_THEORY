\chapter{Prinsip dan Aksioma Matematika Idris (PAMI)}
\label{app:PAMI}

Lampiran ini memuat secara lengkap dan final seluruh fondasi matematis-fisik 
Teori Idris sesuai dokumen tangan 22 November 2025 halaman 1--2.

%%%%%%%%%%%%%%%%%%%%%%%%%%%%%%%%%%%%%%%%%%%%%%%%%%%%%%%%%%%%%
\section{Aksioma Dasar (A1--A4)}
%%%%%%%%%%%%%%%%%%%%%%%%%%%%%%%%%%%%%%%%%%%%%%%%%%%%%%%%%%%%%

\begin{axiom}[A1 — PSI: Prinsip Supremasi Informasi]
Segala sesuatu dalam alam semesta berasal dari struktur informasi dasar. 
Tidak ada ruang, waktu, energi, ataupun materi tanpa informasi.
\end{axiom}

\begin{axiom}[A2 — Atom Informasi: Driston]
Unit terkecil informasi adalah driston, direpresentasikan oleh graf Ramanujan–Idris 
RJI--$N$ reguler derajat-3 dengan batas spektral Ramanujan.
\end{axiom}

\begin{axiom}[A3 — Hukum Kekekalan Informasi Lintas-Era]
\begin{equation}
N_{\rm Drissian} = N_{\rm Planck} = N_{\rm Emergen}
\end{equation}
Jumlah driston kekal sepanjang semua fase kosmik.
\end{axiom}

\begin{axiom}[A4 — Aksi Dasar SD]
Aksi minimal pada fase SD (Strong Drissian) adalah
\begin{equation}
S_D = \sum_{ij} A_{ij} I_i I_j
\end{equation}
di mana $A_{ij}$ adalah matriks adjacency graf RJI--$N$.
\end{axiom}

%%%%%%%%%%%%%%%%%%%%%%%%%%%%%%%%%%%%%%%%%%%%%%%%%%%%%%%%%%%%%
\section{Definisi Fundamental Idrissian (D1--D6)}
%%%%%%%%%%%%%%%%%%%%%%%%%%%%%%%%%%%%%%%%%%%%%%%%%%%%%%%%%%%%%

\begin{definition}[D1 — Driston Objek]
Driston adalah titik graf RJI--$N$ yang merepresentasikan 
unit informasi terkecil.
\end{definition}

\begin{definition}[D2 — Graf Ramanujan–Idris RJI--$N$]
Graf reguler derajat-3 dengan $N$ titik yang memenuhi 
batas spektral Ramanujan:
\begin{equation}
|\lambda_k| \leq 2\sqrt{2} \quad \forall k \geq 1
\end{equation}
\end{definition}

\begin{definition}[D3 — Operator Informasi Dasar $L_I$]
\begin{equation}
L_I = 3I - \frac{2}{3}A
\end{equation}
\end{definition}

\begin{definition}[D4 — Ruang Hilbert Informasional $\mathcal{H}_I$]
\begin{equation}
\mathcal{H}_I = \operatorname{span}_\mathbb{C} \{ |\psi_k\rangle \}_{k=0}^{N-1},
\qquad
L_I |\psi_k\rangle = \lambda_k |\psi_k\rangle
\end{equation}
Semua mode real, orthonormal, spektrum diskrit.
\end{definition}

\begin{definition}[D5 — Informational Renormalization Group (IRG)]
Flow skala spektral:
\begin{equation}
\lambda_k(N) \to \lambda_k(\infty) + \mathcal{O}(N^{-1/2})
\end{equation}
\end{definition}

\begin{definition}[D6 — ICP: Informational Cosmological Principle]
Struktur spektral graf RJI--$N$ homogen dan isotrop pada skala besar, 
sehingga kosmologi efektif adalah FLRW dengan $\Omega_{\rm tot}=1$ secara otomatis.
\end{definition}

%%%%%%%%%%%%%%%%%%%%%%%%%%%%%%%%%%%%%%%%%%%%%%%%%%%%%%%%%%%%%
\section{Prediksi Fisik Fundamental (P1--P4)}
%%%%%%%%%%%%%%%%%%%%%%%%%%%%%%%%%%%%%%%%%%%%%%%%%%%%%%%%%%%%%

\newtheorem{prediction}{Prediction}[chapter]

\begin{prediction}[P1 --- Spektrum Massa Partikel SM]
Untuk semua mode fermion/boson:
\begin{equation}
m_k \propto \lambda_k \qquad \Rightarrow \qquad 
m_k = \alpha_k \sqrt{\lambda_k}
\end{equation}
\end{prediction}

\begin{prediction}[P2 --- Massa Fisik]
\begin{equation}
m_{\rm phys} = \alpha_k \frac{E_k}{c^2}
\end{equation}
\end{prediction}

\begin{prediction}[P3 --- Dark Matter Idrissian (IDM)]
Mode spektral dengan $\lambda_k \in (0.3, 1.2)$ 
menghasilkan materi gelap non-baryonik, $\Omega_{\rm CDM} \approx 0.27$.
\end{prediction}

\begin{prediction}[P4 --- Idrissian Dark Energy (IDE)]
Mode spektral dengan $\lambda_k > 1.2$ 
menghasilkan energi gelap kosmologis, $\Omega_\Lambda \approx 0.68$.
\end{prediction}

%%%%%%%%%%%%%%%%%%%%%%%%%%%%%%%%%%%%%%%%%%%%%%%%%%%%%%%%%%%%%
\section{Ringkasan Struktur Teori Idris}
%%%%%%%%%%%%%%%%%%%%%%%%%%%%%%%%%%%%%%%%%%%%%%%%%%%%%%%%%%%%%

\begin{tabular}{ll}
A1 & PSI — Supremasi Prinsip Informasi \\
A2 & Driston (atom informasi) \\
A3 & Kekekalan $N$ lintas-era \\
A4 & Aksi Dasar SD: $S_D = \sum A_{ij}I_iI_j$ \\
D1--D6 & Definisi graf, $L_I$, $\mathcal{H}_I$, IRG, ICP \\
P1--P4 & Prediksi massa SM, dark matter, dark energy \\
\end{tabular}

Semua fisika (ruang-waktu, gravitasi, partikel SM, gaya-gaya, kosmologi, 
dark matter, dark energy, gaya kelima, multiverse) 
adalah konsekuensi matematis langsung dari satu graf RJI--$N$ 
dan satu operator $L_I = 3I - \frac{2}{3}A$ dalam limit kontinuum.

Tidak ada postulat tambahan.  
Tidak ada parameter bebas.  
Tidak ada numerologi.

\qed