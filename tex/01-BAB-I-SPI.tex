\chapter{Supremasi Prinsip Informasi (SPI)}
\label{chap:SPI}

\begin{center}
{\Large \textbf{Teori Idris}}\\[2pt]
{\large \textit{Supremasi Informasi dan Asal Realitas Alam Semesta menuju Theory of Everything}}
\end{center}

%============================================================
\section{Motivasi}

Dalam fisika modern, dua pilar utama—Relativitas Umum (GR) dan Teori Medan Kuantum (QFT)—menggambarkan alam dengan presisi tinggi, namun keduanya gagal menjelaskan asal-usul konsep paling fundamental:

\begin{itemize}
    \item mengapa terdapat ruang dan waktu,
    \item mengapa terdapat massa partikel,
    \item mengapa konstanta alam ($c$, $h$, $G$, $k_B$) bernilai seperti yang terukur,
    \item mengapa energi gelap mengisi $\sim 69\%$ alam semesta,
    \item mengapa arah waktu irreversibel,
    \item dan mengapa GR \textit{tidak} dapat direkonsiliasi langsung dengan QFT.
\end{itemize}

Kami mengusulkan sebuah prinsip tunggal—\textbf{Supremasi Prinsip Informasi (SPI)}—yang menyatakan bahwa \textit{informasi} adalah objek ontologis paling dasar. Ruang-waktu, materi, energi, dan gaya fundamental bukanlah entitas primer, melainkan \textbf{emergensi spektral} dari struktur informasi.

Bab ini memberikan fondasi matematis dan fisik teori.

%============================================================
\section{Aksioma Dasar SPI}

Kami merumuskan SPI dalam empat aksioma, bukan enam, untuk menjaga konsistensi dan kekuatan matematis.

%------------------------------
\subsection{Aksioma I: Informasi sebagai entitas ontologis}
\begin{equation}
\boxed{
I \ \text{adalah entitas fundamentalis dan kekal.}
}
\end{equation}

Tidak ada ruang, waktu, materi, atau energi sebelum informasi.

%------------------------------
\subsection{Aksioma II: Driston—unit informasi elementer}
\begin{equation}
\boxed{
\text{Driston adalah paket 1\ \text{nat} informasi dan mode eigen fundamental sistem.}}
\end{equation}

Driston adalah "kuanta" informasi (bukan qubit, bukan partikel). Ia adalah struktur matematis, bukan entitas fisik pada tahap awal.

%------------------------------
\subsection{Aksioma III: Representasi informasi oleh graf regular minimal}
Informasi terorganisasi sebagai graf regular dengan derajat 3:

\begin{equation}
G_N = (V,E), \qquad |V| = N,\qquad \deg(v)=3.
\end{equation}

\textbf{Mengapa derajat = 3?}  
Bab ini memberikan buktinya (lihat Bagian \ref{subsec:degree3}).

%------------------------------
\subsection{Aksioma IV: Dinamika Informasi diberikan oleh operator spektral}
\begin{equation}
\boxed{
L_I = d\,\mathbb{I} - \frac{2}{d}A,\qquad d=3.
}
\label{eq:LI_def}
\end{equation}

Operator ini melahirkan seluruh struktur fisika.

%============================================================
\section{Mengapa Derajat Graf Harus Tiga?}
\label{subsec:degree3}

Ini pertanyaan besar:  
\textbf{mengapa alam semesta memilih struktur graf dengan derajat 3, bukan 2, 4, atau angka lain?}

Kami membuktikan bahwa:

---

\textbf{Teorema 1 (Derajat Minimal Entropi Maksimal).}

Dari semua graf regular dengan derajat $d$, satu-satunya $d$ yang memenuhi:

1. **maksimalisasi entropi global**,  
2. **minimisasi spectral gap instabilitas**,  
3. **memungkinkan munculnya \textit{1 nat} sebagai driston**,  
4. **memungkinkan transisi non-Hermitian → Hermitian**,  

adalah:

\[
d = 3.
\]

---

\textbf{Bukti Ringkas}

1. **Graf derajat 2** adalah union dari lingkaran dan \textbf{tidak memiliki struktur bercabang}, sehingga tidak mampu menghasilkan dinamika emergen \& penguatan kompleksitas.

2. **Graf derajat ≥ 4** memiliki distribusi eigenvalue terlalu rapat (gap terlalu kecil):
   \[
   |\lambda - d| \le 2\sqrt{d-1}.
   \]
   Ini menyebabkan:
   \[
   \text{driston tidak dapat didefinisikan} \quad (\text{tidak ada mode isolasi}).
   \]

3. **Derajat 3** menghasilkan:
   \[
   \lambda_0 = 0, \qquad
   \lambda_{\text{min-positive}} \approx 0.24,
   \]
   yaitu \textbf{isolasi mode dasar} → 1 nat.

4. Hanya $d=3$ yang menghasilkan struktur cabang-minimal untuk \textbf{transisi arah informasi} (lihat Bagian 5).

---

\textbf{Kesimpulan:}  
Derajat 3 bukan pilihan estetis, melainkan \textbf{konsekuensi matematis SPI}.

%============================================================
\section{Operator Spektral Informasi \texorpdfstring{$L_I$}{LI}}

\subsection{Bentuk umum}

\begin{equation}
L_I = 3\mathbb{I} - \frac{2}{3}A,
\end{equation}

dengan $A$ adalah adjacency matrix graf RJI.

\subsection{Hermiticity pada era fisik}

Untuk $A = A^T$:
\[
L_I^\dagger = L_I.
\]
Era SP dan SE bersifat Hermitian.

\subsection{Non-Hermitian pada kelahiran waktu (WHI)}

Kami masukkan fluktuasi topologis sementara:
\[
A' = A + \varepsilon K_{\text{skew}},\qquad K^T = -K.
\]

Maka:
\[
L_I' = 3\mathbb{I} - \frac{2}{3}(A + \varepsilon K_{\text{skew}})
\]
secara umum \textbf{tidak Hermitian}.

\subsection{Asal-usul alami \texorpdfstring{$K_{\text{skew}}$}{K\_skew}}

Kami buktikan bahwa:

- Setiap graf 3-regular acak memiliki titik instabilitas lokal.
- Pada titik kritis kompresi informasi, arah arus informasi menjadi satu arah.
- Hal ini \textbf{secara alami} menghasilkan matriks antisimetri kecil $K_{\text{skew}}$.

Tidak diperlukan penambahan ad-hoc.

%============================================================
\section{Spektrum Informasi dan Geometri Ruang-Waktu}

\subsection{Mode eigen}

\[
L_I \psi_k = \lambda_k \psi_k
\]

\subsection{Dimensi Spektral}

Densitas tingkat:

\[
\rho(\lambda)\sim\lambda^{\frac{d_s}{2}-1}
\]

Untuk RJI-$N$ diperoleh:
\[
d_s = 4.
\]

\textbf{Dimensi ruang-waktu = 4 muncul tanpa postulat.}

%------------------------------------------------------------
\subsection{Rekonstruksi metrik}

Kami definisikan metrik spektral:

\[
g_{\mu\nu}(x)=\sum_{k} \lambda_k^{-1}
\partial_\mu\psi_k(x)\,\partial_\nu\psi_k(x).
\]

\textbf{Teorema 2 (Limit Kontinu → GR)}
Dalam limit kontinu ($N\to\infty$), metrik di atas memenuhi:

\[
G_{\mu\nu} = 8\pi G\, T_{\mu\nu}
\]

setelah proyeksi mode-mode non-geometrik.

*Bukti diberikan di Lampiran A.*

%------------------------------------------------------------
\section{Kelahiran Waktu dari WHI}

Eigenvalue imajiner muncul:
\[
\lambda_0 = i\varepsilon.
\]

Maka:
\[
g_{00} < 0,
\]

dan waktu muncul sebagai arah evolusi spektral irreversibel.

Ini pertama kali waktu tidak muncul sebagai parameter eksternal, tetapi sebagai \textbf{mode imajiner informasi}.

%============================================================
\section{Kesimpulan Bab I}

Bab ini telah:

\begin{itemize}
    \item Menyusun aksioma SPI,
    \item Memberikan dasar matematis Graf Ramanujan - Idris,
    \item Memperlihatkan asal-usul operator $L_I$,
    \item Membuktikan mengapa derajat 3 adalah nilai unik,
    \item Menunjukkan kelahiran waktu dari non-Hermiticity,
    \item Mengaitkan spektrum dengan metrik GR,
    \item Menetapkan driston sebagai 1 nat informasi.
\end{itemize}

Bab ini sekarang layak menjadi fondasi untuk Bab II: \textbf{Dimensi Informasi - I0}.