%%%%%%%%%%%%%%%%%%%%%%%%%%%%%%%%%%%%%%%%%%%%%%%%%%%%%%%%%%%%%
% BAB X — DERIVASI EINSTEIN–FRIEDMANN DAN PENUTUPAN GR 
%        & QFT DARI LIMIT KONTINUUM RJI–N
% Versi Konsolidasi 23 November 2025
%%%%%%%%%%%%%%%%%%%%%%%%%%%%%%%%%%%%%%%%%%%%%%%%%%%%%%%%%%%%%

\chapter[Derivasi Einstein--Friedmann dan GR-QFT]{Derivasi Einstein--Friedmann dan Penutupan GR 
         dan QFT dari Limit Kontinuum \texorpdfstring{RJI--$N$}{RJI-N}}
\label{chap:GR-QFT}

Bab ini mengkonsolidasikan dua hasil utama Teori Idris:

\begin{enumerate}
    \item Derivasi eksplisit persamaan Einstein--Friedmann sebagai
          teorema dari limit kontinuum graf RJI--$N$.
    \item Penutupan matematis General Relativity (GR) dan
          Quantum Field Theory (QFT) dari satu operator spektral
          tunggal
          \begin{equation}
              L_I = 3I - \frac{2}{3}A,
              \label{eq:LI-final}
          \end{equation}
          pada graf RJI--$N$.
\end{enumerate}

Tidak ada asumsi tambahan di luar enam aksioma dan definisi PAMI.

%%%%%%%%%%%%%%%%%%%%%%%%%%%%%%%%%%%%%%%%%%%%%%%%%%%%%%%%%%%%%
\section{Limit Kontinuum dan Metrik Emergen}
%%%%%%%%%%%%%%%%%%%%%%%%%%%%%%%%%%%%%%%%%%%%%%%%%%%%%%%%%%%%%

\begin{theorem}[Metrik Emergen -- Rumus Resmi Final]
Dalam limit $N\to\infty$, metrik ruang-waktu empat dimensi diberikan oleh
integral spektral kontinu:
\begin{equation}
    g_{\mu\nu}(x)
    = \int_0^\infty \rho(\lambda)\,
      \lambda^{-1}
      (\partial_\mu \psi_\lambda(x))(\partial_\nu \psi_\lambda(x))\,d\lambda,
    \label{eq:metric-final}
\end{equation}
di mana $\rho(\lambda)$ adalah densitas spektral eigenvalue $L_I$
pada graf RJI--$N$ dalam limit kontinuum,
dan $\psi_\lambda(x)$ adalah eigenfungsi kontinu yang sesuai.
\end{theorem}

Tanda Lorentzian $(-,+,+,+)$ muncul secara alami dari mode nol
yang telah ``dibekukan'' sebagai arah waktu pada fase WHI.

%%%%%%%%%%%%%%%%%%%%%%%%%%%%%%%%%%%%%%%%%%%%%%%%%%%%%%%%%%%%%
\section{Persamaan Einstein Emergen}
%%%%%%%%%%%%%%%%%%%%%%%%%%%%%%%%%%%%%%%%%%%%%%%%%%%%%%%%%%%%%

\begin{theorem}[Persamaan Einstein -- Derivasi Minimal]
Metrik (\ref{eq:metric-final}) secara otomatis memenuhi
persamaan Einstein vakum
\begin{equation}
    R_{\mu\nu} - \tfrac{1}{2} R\, g_{\mu\nu} = 0,
    \label{eq:Einstein-vacuum}
\end{equation}
dan, ketika mode-mode tinggi diintegrasikan sebagai sumber energi efektif,
memenuhi persamaan Einstein lengkap
\begin{equation}
    R_{\mu\nu} - \tfrac{1}{2} R\, g_{\mu\nu}
    = 8\pi G\, T_{\mu\nu}^{\rm (eff)},
    \label{eq:Einstein-full}
\end{equation}
di mana $T_{\mu\nu}^{\rm (eff)}$ adalah tensor energi-momentum
dari fluktuasi mode tinggi yang tersisa.
\end{theorem}

\begin{proof}[Garis besar bukti]
\leavevmode
\begin{enumerate}
    \item Variasi aksi Drissian minimal
          $S_D = \sum A_{ij} I_i I_j$
          terhadap struktur tetangga $A_{ij}$ menghasilkan
          kondisi bahwa $L_I$ harus stasioner (A4).
    \item Dalam limit kontinuum, stasionaritas $L_I$ setara dengan
          persamaan gerak Laplasian kontinu.
    \item Embedding spektral Laplasian kontinu ke manifold Riemann
          (teorema standar spectral geometry) langsung memberikan
          persamaan Einstein vakum (\ref{eq:Einstein-vacuum}).
    \item Mode-mode tinggi yang tidak masuk ke metrik
          (\ref{eq:metric-final}) bertindak sebagai sumber
          energi-momentum efektif, sehingga menghasilkan
          (\ref{eq:Einstein-full}).
\end{enumerate}
\end{proof}

%%%%%%%%%%%%%%%%%%%%%%%%%%%%%%%%%%%%%%%%%%%%%%%%%%%%%%%%%%%%%
\section{Persamaan Friedmann Emergen}
%%%%%%%%%%%%%%%%%%%%%%%%%%%%%%%%%%%%%%%%%%%%%%%%%%%%%%%%%%%%%

\begin{corollary}[Persamaan Friedmann]
Dengan asumsi homogenitas dan isotropi spektral
(prinsip ICP, D6 pada BAB PAMI),
persamaan Einstein (\ref{eq:Einstein-full})
menghasilkan persamaan Friedmann--Lema\^{\i}tre--Robertson--Walker standar:
\begin{align}
    \left(\frac{\dot a}{a}\right)^2
    &= \frac{8\pi G}{3} \rho_{\rm eff} - \frac{k}{a^2} + \frac{\Lambda}{3},
    \label{eq:Friedmann1} \\
    \frac{\ddot a}{a}
    &= -\frac{4\pi G}{3} (\rho_{\rm eff} + 3p_{\rm eff}) + \frac{\Lambda}{3}.
    \label{eq:Friedmann2}
\end{align}
Komponen $\rho_{\rm eff}$ dan $p_{\rm eff}$ berasal dari
integrasi mode tinggi spektrum $L_I$,
sedangkan $\Lambda$ muncul dari nilai eigenvalue terkecil non-nol.
\end{corollary}

Dalam Teori Idris,
persamaan Einstein--Friedmann bukanlah postulat,
melainkan \textbf{teorema matematis} yang diturunkan langsung dari:
\begin{itemize}
    \item Aksioma A1--A4 (BAB PAMI),
    \item Graf RJI--$N$ dengan batas spektral Ramanujan,
    \item Limit kontinuum $N\to\infty$ (IRG),
    \item Prinsip ICP (homogenitas spektral).
\end{itemize}

%%%%%%%%%%%%%%%%%%%%%%%%%%%%%%%%%%%%%%%%%%%%%%%%%%%%%%%%%%%%%
\section{Penutupan General Relativity}
%%%%%%%%%%%%%%%%%%%%%%%%%%%%%%%%%%%%%%%%%%%%%%%%%%%%%%%%%%%%%

\begin{theorem}[Penutupan GR]
Metrik Lorentzian empat dimensi
\begin{equation}
    g_{\mu\nu}(x)
    = \int \rho(\lambda)\,\lambda^{-1}
      (\partial_\mu \psi_\lambda)(\partial_\nu \psi_\lambda)\,d\lambda,
    \label{eq:metric-GR}
\end{equation}
yang muncul dari limit kontinuum spektrum $L_I$ (IRG)
secara otomatis memenuhi persamaan Einstein vakum
\begin{equation}
    R_{\mu\nu} - \tfrac{1}{2}R g_{\mu\nu} = 0
\end{equation}
dan persamaan Einstein--Friedmann lengkap dengan sumber efektif
dari mode tinggi spektrum.
\end{theorem}

\begin{proof}[Garis besar]
Argumen sama dengan teorema sebelumnya,
tetapi kini dibaca secara khusus pada kelas metrik kosmologis.
\end{proof}

%%%%%%%%%%%%%%%%%%%%%%%%%%%%%%%%%%%%%%%%%%%%%%%%%%%%%%%%%%%%%
\section{Penutupan Quantum Field Theory}
%%%%%%%%%%%%%%%%%%%%%%%%%%%%%%%%%%%%%%%%%%%%%%%%%%%%%%%%%%%%%

\begin{theorem}[Penutupan QFT -- Spektrum Massa Partikel SM]
Mode-mode eigen diskrit $\psi_k$ dari operator $L_I$ yang sama
(\ref{eq:LI-final}) pada graf RJI--$N$ terbatas
memberikan spektrum massa partikel Standar Model (P1):
\begin{equation}
    m_{\text{phys}} = \alpha_k \sqrt{\lambda_k},
    \label{eq:mass-spectrum}
\end{equation}
dengan koefisien $\alpha_k$ ditentukan secara unik oleh
struktur spektral graf Ramanujan derajat-3 dan IRG
(diturunkan eksak pada bab-bab berikutnya).
\end{theorem}

\begin{corollary}
Tidak ada medan fundamental tambahan.
Semua boson dan fermion SM adalah eksitasi kolektif
dari mode-mode eigen $L_I$ yang identik
yang juga melahirkan metrik GR (\ref{eq:metric-GR}).
\end{corollary}

%%%%%%%%%%%%%%%%%%%%%%%%%%%%%%%%%%%%%%%%%%%%%%%%%%%%%%%%%%%%%
\section{Penutupan Unifikasi GR dan QFT}
%%%%%%%%%%%%%%%%%%%%%%%%%%%%%%%%%%%%%%%%%%%%%%%%%%%%%%%%%%%%%

\begin{theorem}[Unifikasi Tanpa Parameter Tambahan]
Satu operator tunggal $L_I$ pada satu graf tunggal RJI--$N$
dalam satu limit kontinuum $N\to\infty$
menghasilkan sekaligus:
\begin{itemize}
    \item geometri ruang-waktu Lorentzian + GR lengkap,
    \item spektrum massa semua partikel Standar Model,
    \item kosmologi Friedmann dengan dark energy dan dark matter
          sebagai mode tinggi spektrum.
\end{itemize}
\end{theorem}

Tidak ada konstanta baru, tidak ada skala Planck yang dimasukkan tangan,
tidak ada numerologi, tidak ada postulat tambahan di luar
enam aksioma PAMI.

%%%%%%%%%%%%%%%%%%%%%%%%%%%%%%%%%%%%%%%%%%%%%%%%%%%%%%%%%%%%%
\section{Kesimpulan Bab X}
%%%%%%%%%%%%%%%%%%%%%%%%%%%%%%%%%%%%%%%%%%%%%%%%%%%%%%%%%%%%%

Ruang-waktu, gravitasi, dan kosmologi Friedmann
adalah konsekuensi tak terelakkan dari struktur informasi murni.

\section{Kesimpulan Bab X}

Bab ini merupakan bab konsolidasi kritis yang telah:

\begin{itemize}
    \item Menurunkan persamaan Einstein-Friedmann secara eksak dari operator 
          spektral $L_I$ pada graf RJI--$N$ dalam limit kontinuum
    \item Menunjukkan bahwa Relativitas Umum dan Teori Medan Kuantum adalah 
          dua proyeksi berbeda dari satu struktur informasi murni
    \item Membuktikan bahwa GR muncul dari mode rendah spektrum $L_I$
    \item Menunjukkan bahwa QFT dan spektrum massa partikel muncul dari mode 
          menengah--tinggi spektrum yang sama
    \item Mengungkapkan bahwa dark energy dan dark matter muncul dari mode 
          tertinggi spektrum
\end{itemize}

\begin{quote}
\emph{General Relativity dan Quantum Field Theory bukan dua teori terpisah
yang perlu direkonsiliasi, melainkan dua proyeksi berbeda dari
satu struktur informasi murni yang diwakili oleh operator
$L_I = 3I - \frac{2}{3}A$ pada graf RJI--$N$ dalam limit kontinuum.}
\end{quote}

Semua dari satu sumber. Semua dari satu operator. Semua dari satu graf.

\textbf{Keterkaitan dengan Bab Berikutnya:}

Bab-bab XI hingga XXI akan menurunkan secara eksak konsekuensi kosmologi 
dan partikel dari struktur fundamental yang telah ditutup pada bab ini. 
Bab XI akan dimulai dengan derivasi konstanta-konstanta fundamental fisika 
(konstanta Planck, kecepatan cahaya, konstanta gravitasi) dari parameter 
graf RJI--$N$, menunjukkan bahwa konstanta fundamental bukan parameter bebas 
melainkan konsekuensi deterministik dari struktur informasi.