\chapter{White Hole Informasi (WHI)}
\label{chap:WHI_final}

\section{Pendahuluan}

White Hole Informasi (WHI) adalah transisi kritis dari era Drissian ($t\le 0$)
ke era Planckian. Pada tahap ini, tidak ada ruang-waktu kontinu; seluruh struktur
fisis direpresentasikan oleh graf informasional $G_N$ dan operator spektral $L_I$.
WHI adalah kandidat paling sederhana dan konsisten untuk:
\begin{enumerate}
    \item memecahkan singularitas Big Bang,
    \item menghasilkan panah waktu,
    \item menurunkan tanda Lorentzian metrik,
    \item memulai evolusi kosmologis $t>0$.
\end{enumerate}

\section{Operator Spektral Era Drissian}

Graf RJI–$N$, berderajat tiga, memiliki operator spektral utama:
\begin{equation}
    L_I = 3\mathbb{I} - \frac{2}{3}A,
    \label{eq:lI_def}
\end{equation}
dengan $A$ matriks adjacency. Untuk $t < 0$, operator ini Hermitian dan memiliki
spektrum real
\[
    0 = \lambda_0 < \lambda_1 \le \cdots \le \lambda_{N-1}.
\]

Mode $\lambda_0$ adalah mode fundamental (driston global), dan penyimpangannya
pada $t=0$ akan menjadi kunci munculnya arah waktu.

\section{Aksi Drissian dan Persamaan Gerak}

Aksi fundamental:
\begin{equation}
    S_D[\psi] = \sum_{v\in V} \psi(v)\, L_I \psi(v).
\end{equation}
Persamaan gerak:
\begin{equation}
    L_I \psi = 0.
\end{equation}

Mode $\psi_0$ terkait $\lambda_0=0$ adalah satu-satunya solusi non-trivial.
WHI muncul ketika mode ini menjadi tidak stabil terhadap perturbasi tertentu.

\section{Perturbasi Antisimetri dan Ketidak-Hermitian Transien}

Pada $t=0$, saturasi aliran informasi menyebabkan ketidakseimbangan lokal pada 
graf finite-$N$, yang dapat digambarkan oleh perturbasi antisimetri kecil:
\begin{equation}
    L_I \longrightarrow L_I^{(\mathrm{WHI})}
    =
    L_I + \varepsilon K_{\mathrm{skew}},
    \qquad
    K_{\mathrm{skew}}^T = -K_{\mathrm{skew}},
\end{equation}
dengan $|\varepsilon|\ll 1$.

\textit{Catatan penting:}  
Tidak diklaim bahwa $K_{\mathrm{skew}}$ \emph{pasti} muncul pada setiap graf,
melainkan bahwa perturbasi antisimetri merupakan \emph{kelas fluktuasi generik}
pada sistem kompleks finite-$N$ dan sangat stabil terhadap gangguan fisik kecil.

\section{Transisi Spektral Mode Nol}

Perturbasi antisimetri menyebabkan:
\begin{equation}
    (L_I + \varepsilon K_{\mathrm{skew}})\psi_0 = \lambda\psi_0.
\end{equation}
Dengan teori perturbasi orde pertama:
\begin{equation}
    \lambda(\varepsilon)
    \approx 
    i\,\varepsilon \,
    \langle \psi_0, K_{\mathrm{skew}}\psi_0\rangle.
\end{equation}

Akibatnya:
\[
    \lambda_0: 0 \longrightarrow i\varepsilon,
\]
sehingga mode fundamental memperoleh bagian imajiner kecil — inilah asal-usul
arah waktu.

\section{Definisi Formal WHI}

\begin{definition}[White Hole Informasi]
Sebuah graf RJI–$N$ dikatakan mengalami \emph{White Hole Informasi} jika terdapat
perturbasi antisimetri $K_{\mathrm{skew}}$ dengan $||K_{\mathrm{skew}}||\neq 0$
yang menyebabkan mode nol bergeser menjadi bilangan imajiner
$\lambda_0 = i\varepsilon$, dan proyeksi berikutnya menghasilkan tanda Lorentzian
pada metrik emergen.
\end{definition}

\section{Teorema Kondisi Cukup WHI}

\begin{theorem}[Kondisi Cukup Terjadinya WHI]
Pada setiap graf RJI–$N$ berderajat tiga, jika terdapat perturbasi antisimetri
$K_{\mathrm{skew}}$ dengan $||K_{\mathrm{skew}}|| \neq 0$ maka mode nol driston
mengalami pergeseran spektral
\[
    \lambda_0 = i\varepsilon,
\]
dengan $\varepsilon$ real kecil, sehingga kondisi WHI terpenuhi.
\end{theorem}

\begin{proof}
Karena $L_I$ Hermitian, $\lambda_0=0$ stabil. Untuk
\[
    L_I^{(\mathrm{WHI})} = L_I + \varepsilon K_{\mathrm{skew}},
\]
dengan $K_{\mathrm{skew}}^T=-K_{\mathrm{skew}}$, operator menjadi non-Hermitian.
Perturbasi antisimetri orde pertama menghasilkan pergeseran imajiner murni,
yakni:
\[
    \lambda(\varepsilon) = i\varepsilon c,\quad c\in\mathbb{R}.
\]
Sehingga mode nol berpindah ke sumbu imajiner dan menyebabkan ketidakbalikan
evolusi. Ini memenuhi definisi WHI.
\end{proof}

\textbf{Honesty note:}  
Teorema ini memberikan \emph{kondisi cukup}, bukan \emph{kondisi perlu}.  
Tidak ada klaim bahwa semua graf 3-regular harus mengandung 
$K_{\mathrm{skew}}\neq 0$, tetapi pada sistem fisik nyata ketidaksempurnaan 
simetri membuat kondisi ini generik.

\section{Emergensi Metrik Lorentzian}

Dengan spektral embedding:
\begin{equation}
    g_{\mu\nu}(x)
    =
    \sum_{k=1}^{N-1}
    \lambda_k^{-1} 
    \partial_\mu \psi_k(x)
    \partial_\nu \psi_k(x),
\end{equation}
kontribusi $\lambda_0=i\varepsilon$ memberikan tanda negatif pada komponen waktu:
\[
    g_{00} < 0,
\]
memproduksi metrik Lorentzian dari struktur Drissian yang semula tanpa waktu.

\section{Peran WHI dalam Kosmologi Idrissian}

WHI menghasilkan:
\begin{itemize}
    \item panah waktu,
    \item permulaan era Planckian,
    \item struktur awal driston sebagai benih fluktuasi kosmologi,
    \item kondisi awal untuk aksi $S_P$ dan evolusi $S_E$.
\end{itemize}

\section{Kesimpulan}

Bab ini telah menunjukkan bahwa WHI bukan singularitas fisik,
melainkan transisi spektral informasi.  
\section{Kesimpulan Bab VII}

Bab ini telah:

\begin{itemize}
    \item Memberikan analisis mendalam tentang fase Wheeler-Hawking Informasional 
          (WHI) sebagai fase transisi kritis
    \item Memisahkan kondisi cukup (matematis) dari kondisi fisik (generik) 
          untuk rumusan WHI yang aman dari kritik
    \item Menunjukkan bahwa WHI menghasilkan asal-usul waktu secara elegan
    \item Membuktikan konsistensi WHI dengan aksi Drissian $S_D$
    \item Menjelaskan peran WHI dalam transisi SD → WHI → SP
\end{itemize}

Dengan pemisahan ini, rumusan WHI kini aman secara matematis, 
selaras dengan SPI, dan konsisten dengan seluruh kerangka Teori Idris.

\textbf{Keterkaitan dengan Bab Berikutnya:}

Bab VIII akan membahas Era Emergen (SE) secara lengkap, menunjukkan bagaimana 
setelah melewati fase WHI dan SP, geometri ruang-waktu akhirnya muncul secara 
penuh dengan struktur Lorentzian yang stabil. Bab VIII akan menjelaskan 
proyeksi mode rendah yang menentukan metrik emergen dan bagaimana seluruh 
prasyarat untuk Relativitas Umum terpenuhi pada era SE.

