%%%%%%%%%%%%%%%%%%%%%%%%%%%%%%%%%%%%%%%%%%%%%%%%%%%%%%%%%%%%%
% BAB XXI — DINAMIKA IDE DAN NILAI w_IDE
% Teori Idris — Versi LaTeX Lengkap
%%%%%%%%%%%%%%%%%%%%%%%%%%%%%%%%%%%%%%%%%%%%%%%%%%%%%%%%%%%%%

\chapter[Dinamika Energi Gelap Idrissian]{Dinamika Energi Gelap Idrissian dan Penurunan Nilai \texorpdfstring{$w_{\rm IDE}$}{w\_IDE}}
\label{chap:IDE-dynamics}

Energi Gelap Idrissian (IDE) muncul langsung dari mode-mode informasi
dengan nilai eigen $\lambda > 1.2$ pada operator informasi
\begin{equation}
L_I = 3I - \frac{2}{3}A.
\label{eq:LI-def-IDE}
\end{equation}
IDE bukanlah medan tambahan atau parameter kosmologis ad-hoc.
Bab ini menurunkan dinamika IDE, karakter tekanan-negatifnya, 
nilai $w_{\rm IDE}$, dan prediksi waktu Kiamat Idrissian.

%%%%%%%%%%%%%%%%%%%%%%%%%%%%%%%%%%%%%%%%%%%%%%%%%%%%%%%%%%%%%
\section{IDE sebagai Mode Spektral Tinggi}
%%%%%%%%%%%%%%%%%%%%%%%%%%%%%%%%%%%%%%%%%%%%%%%%%%%%%%%%%%%%%

Setiap mode eigen $L_I$ dengan $\lambda_k > 1.2$ berkontribusi pada 
densitas energi efektif:
\begin{equation}
\rho_k = \frac{1}{2}\lambda_k |c_k|^2.
\label{eq:rho-k-IDE}
\end{equation}

Total energi gelap:
\begin{equation}
\rho_{\rm IDE}
=
\sum_{\lambda_k > 1.2}
\frac{1}{2}\lambda_k |c_k|^2.
\label{eq:rho-IDE}
\end{equation}

Evolusi mode dikendalikan oleh aliran IRG:
\begin{equation}
\frac{dc_k}{d\tau}=-\lambda_k c_k,
\qquad
c_k(\tau)=c_k(0)e^{-\lambda_k\tau}.
\label{eq:IRG-evol}
\end{equation}

%%%%%%%%%%%%%%%%%%%%%%%%%%%%%%%%%%%%%%%%%%%%%%%%%%%%%%%%%%%%%
\section{Peran Mode Tinggi dalam Geometri}
%%%%%%%%%%%%%%%%%%%%%%%%%%%%%%%%%%%%%%%%%%%%%%%%%%%%%%%%%%%%%

Metrik emergen ruang-waktu berasal dari embedding spektral:
\begin{equation}
g_{\mu\nu}(x)
=
\sum_{\lambda_k \le K}
\lambda_k^{-1}
\partial_\mu\psi_k(x)\,
\partial_\nu\psi_k(x).
\label{eq:metric-embed}
\end{equation}

Mode dengan $\lambda > 1.2$ menghasilkan tekanan efektif negatif,
karena kontribusi ruang dan waktunya memiliki tanda berlawanan
dalam Lagrangian efektif:
\begin{equation}
\mathcal{L}_{\rm IDE}
\sim
\sum_{\lambda>1.2}
\left(
\lambda^{-1}(\partial\psi)^2
-
\lambda|c|^2
\right).
\label{eq:L-IDE}
\end{equation}

%%%%%%%%%%%%%%%%%%%%%%%%%%%%%%%%%%%%%%%%%%%%%%%%%%%%%%%%%%%%%
\section{Rumus Tekanan dan Energi IDE}
%%%%%%%%%%%%%%%%%%%%%%%%%%%%%%%%%%%%%%%%%%%%%%%%%%%%%%%%%%%%%

Definisi:
\begin{equation}
p_{\rm IDE} = -\frac{\partial\mathcal{L}}{\partial g_{ii}},
\qquad
\rho_{\rm IDE} = \frac{\partial\mathcal{L}}{\partial g_{00}}.
\label{eq:IDE-p-rho}
\end{equation}

Dari (\ref{eq:L-IDE}), diperoleh struktur umum keadaan:
\begin{equation}
w_{\rm IDE}
=
\frac{p_{\rm IDE}}{\rho_{\rm IDE}}
=
-1 - \Delta,
\label{eq:w-IDE}
\end{equation}
dengan
\begin{equation}
\Delta
=
\frac{
\sum_{\lambda>1.2}
\lambda^{-1}(\partial\psi)^2
}{
\sum_{\lambda>1.2}
\lambda |c|^2
}.
\label{eq:Delta-def}
\end{equation}

Gradien mode tinggi memenuhi kira-kira:
\begin{equation}
(\partial\psi)^2 \sim \lambda,
\label{eq:grad-lambda}
\end{equation}
sehingga:
\begin{equation}
\lambda^{-1}(\partial\psi)^2 \sim 1.
\label{eq:grad-approx}
\end{equation}

Karena penyebut (\ref{eq:Delta-def}) besar, maka
\begin{equation}
\Delta \ll 1,\qquad \Delta > 0.
\label{eq:Delta-small}
\end{equation}

Hasil prediksi:
\begin{equation}
w_{\rm IDE}
\simeq
-1.03 \text{ s.d. } -1.08.
\label{eq:w-predicted}
\end{equation}

%%%%%%%%%%%%%%%%%%%%%%%%%%%%%%%%%%%%%%%%%%%%%%%%%%%%%%%%%%%%%
\section{Solusi Skala Faktor dan Big Rip Spektral}
%%%%%%%%%%%%%%%%%%%%%%%%%%%%%%%%%%%%%%%%%%%%%%%%%%%%%%%%%%%%%

Untuk $w<-1$, solusi FRW efektif adalah:
\begin{equation}
a(t)
=
a_0
\left[
1 - \frac{3}{2}(1+w)H_0(t-t_0)
\right]^{
-\frac{2}{3(1+w)}
}.
\label{eq:scale-factor}
\end{equation}

Kiamat (“Big Rip Idrissian”) terjadi saat tanda kurung sama dengan nol:
\begin{equation}
t_{\rm doom} - t_0
=
\frac{2}{3|1+w|H_0}.
\label{eq:t-doom}
\end{equation}

Dengan
\begin{equation}
H_0 \simeq 2.39\times 10^{-18}\ {\rm s}^{-1}
\label{eq:H0}
\end{equation}
dan
\begin{equation}
|1+w|=\Delta,
\label{eq:1+w}
\end{equation}
maka:
\begin{equation}
t_{\rm doom} - t_0
\simeq
\frac{2}{3\Delta}\times 13.3\ {\rm Gyr}.
\label{eq:t-doom-Gyr}
\end{equation}

%%%%%%%%%%%%%%%%%%%%%%%%%%%%%%%%%%%%%%%%%%%%%%%%%%%%%%%%%%%%%
\section{Prediksi Waktu Kiamat Idrissian}
%%%%%%%%%%%%%%%%%%%%%%%%%%%%%%%%%%%%%%%%%%%%%%%%%%%%%%%%%%%%%

Dengan kisaran $\Delta$:
\[
0.03\le\Delta\le0.08,
\]
persamaan (\ref{eq:t-doom-Gyr}) memberi:

\begin{align}
\Delta=0.03 &\Rightarrow t_{\rm doom}-t_0 \simeq 296\ {\rm Gyr}, \\
\Delta=0.05 &\Rightarrow t_{\rm doom}-t_0 \simeq 177\ {\rm Gyr}, \\
\Delta=0.08 &\Rightarrow t_{\rm doom}-t_0 \simeq 110\ {\rm Gyr}.
\end{align}

Prediksi resmi:
\begin{equation}
t_{\rm doom}-t_0 \simeq 110\text{--}300\ {\rm Gyr},
\qquad
\text{nilai sentral } \sim 180\ {\rm Gyr}.
\label{eq:doom-final}
\end{equation}

%%%%%%%%%%%%%%%%%%%%%%%%%%%%%%%%%%%%%%%%%%%%%%%%%%%%%%%%%%%%%
\section{Kesimpulan Bab}
%%%%%%%%%%%%%%%%%%%%%%%%%%%%%%%%%%%%%%%%%%%%%%%%%%%%%%%%%%%%%

\section{Kesimpulan Bab XIX}

Bab ini telah:

\begin{itemize}
    \item Menurunkan dinamika lengkap Energi Gelap Idrissian (IDE) dari 
          mode-mode spektral tinggi $L_I$
    \item Membuktikan bahwa IDE memiliki equation of state $w_{\rm IDE} < -1$ 
          (phantom dark energy) sebagai konsekuensi langsung dari struktur spektral
    \item Menunjukkan bahwa dinamika phantom ini memaksa ekspansi super-ekspansif 
          alam semesta
    \item Memprediksi "Big Rip Spektral" yang akan terjadi sekitar 110--300 
          milyar tahun dari sekarang (nilai sentral $\sim$ 180 Gyr)
\end{itemize}

Energi Gelap Idrissian bukan konstanta kosmologis statis, melainkan memiliki 
dinamika phantom yang akan mengakhiri alam semesta dalam Big Rip Spektral.

\textbf{Keterkaitan dengan Bab Berikutnya:}

Bab XX akan membahas falsifikasi Teori Idris — yaitu bagaimana teori ini dapat 
diuji, diverifikasi, atau dibantah secara matematis maupun eksperimental. 
Bab XX akan dengan jujur memaparkan limitasi matematis teori, titik-titik rentan 
yang dapat mematahkan teori, dan kriteria eksperimental yang dapat memfalsifikasi 
prediksi-prediksi Teori Idris. Ini adalah bukti bahwa Teori Idris adalah teori 
saintifik yang dapat diuji, bukan spekulasi filosofis.
