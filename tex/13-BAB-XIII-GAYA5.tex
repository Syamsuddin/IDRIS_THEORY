%%%%%%%%%%%%%%%%%%%%%%%%%%%%%%%%%%%%%%%%%%%%%%%%%%%%%%%%%%%%%
% BAB XIII — GAYA KE-LIMA IDRISIAN
% Versi 23 November 2025 — Final
%%%%%%%%%%%%%%%%%%%%%%%%%%%%%%%%%%%%%%%%%%%%%%%%%%%%%%%%%%%%%

\chapter[Gaya Kelima Idrissian (16.5th Force)]{Gaya Kelima Idrissian (16.5th Force) dari Mode Spektral 
         Pasca-\textit{Quantum Chromodynamics} (Hadronik)}
\label{chap:force5}

Bab ini menurunkan prediksi eksperimental paling spesifik dari Teori Idris: 
keberadaan \textit{gaya kelima} yang bekerja di atas skala gaya kuat (hadronik) 
namun di bawah ambang mode-mode \textit{dark matter} non-perturbatif.

%%%%%%%%%%%%%%%%%%%%%%%%%%%%%%%%%%%%%%%%%%%%%%%%%%%%%%%%%%%%%
\section{Motivasi Gaya Kelima}
%%%%%%%%%%%%%%%%%%%%%%%%%%%%%%%%%%%%%%%%%%%%%%%%%%%%%%%%%%%%%

Jika empat gaya fundamental muncul dari empat \textit{pita} eigenvalue operator $L_I$:
\begin{align}
\text{gravitasi} &: \text{pita [}\lambda_1, \lambda_4\text{]} && \text{(4 mode geometri)} \\
\text{elektromagnet} &: \text{pita [}\lambda_5, \lambda_{137}\text{]} && \text{(foton-ke-elektron)} \\
\text{lemah} &: \text{pita [}\lambda_{138}, \lambda_{263}\text{]} && \text{(W, Z boson)} \\
\text{kuat} &: \text{pita [}\lambda_{264}, \lambda_{512}\text{]} && \text{(kuark-gluon)}
\end{align}
maka apakah ada pita-pita lain?

%%%%%%%%%%%%%%%%%%%%%%%%%%%%%%%%%%%%%%%%%%%%%%%%%%%%%%%%%%%%%
\section{Prediksi Mode Spektral Pasca-Hadronik}
%%%%%%%%%%%%%%%%%%%%%%%%%%%%%%%%%%%%%%%%%%%%%%%%%%%%%%%%%%%%%

\begin{definition}[Gaya Kelima Idrissian]
Gaya kelima muncul dari pita eigenvalue:
\begin{equation}
\text{gaya-5}: \text{pita [}\lambda_{513}, \lambda_{1024}\text{]}
\end{equation}
yaitu di atas pita gaya kuat (quark \& gluon) 
tetapi di bawah ambang mode-mode yang menjadi dark matter non-perturbatif.
\end{definition}

%%%%%%%%%%%%%%%%%%%%%%%%%%%%%%%%%%%%%%%%%%%%%%%%%%%%%%%%%%%%%
\section{Sifat-Sifat Gaya Kelima}
%%%%%%%%%%%%%%%%%%%%%%%%%%%%%%%%%%%%%%%%%%%%%%%%%%%%%%%%%%%%%

\begin{enumerate}
\item \text{Mediator}: boson vektor atau skalar ringan dengan massa tertentu.\\
\item \text{Jangkauan}: $10^{-15}$\,\text{m} $\lesssim r_5 \lesssim 10^{-18}$\,\text{m}.\\
\item \text{Kopling ke SM}: $g_5 \approx 10^{-3} \to 10^{-6} \cdot g_{\rm weak}$.\\
\item \text{Efek observasional}: penyimpangan kecil pada presisi Z-pole,\\
    deviasi coupling Higgs ringan, dan efek non-standar pada jet lemah (weak-jet).
\end{enumerate}

%%%%%%%%%%%%%%%%%%%%%%%%%%%%%%%%%%%%%%%%%%%%%%%%%%%%%%%%%%%%%
\section{Kesimpulan Bab XIII}
%%%%%%%%%%%%%%%%%%%%%%%%%%%%%%%%%%%%%%%%%%%%%%%%%%%%%%%%%%%%%

Gaya kelima Idrisian:
\begin{enumerate}
\item Merupakan prediksi otomatis dari struktur spektral $L_I$,
\item Bekerja di skala post-hadronik ($\sim 10^{-15}$--$10^{-18}$ m),
\item Hampir tidak berinteraksi dengan SM tetapi terdeteksi secara kuantum,
\item Dapat diuji melalui presisi eksperimen collider berikutnya.
\end{enumerate}

\begin{center}
\boxed{
\text{Gaya ke-5} = \text{pita [}\lambda_{513}, \lambda_{1024}\text{]} \text{ dari } L_I = 3I - \frac{2}{3}A
}
\end{center}

\textbf{Keterkaitan dengan Bab Berikutnya:}

Bab XIV akan melanjutkan analisis spektral dengan menurunkan massa 34 partikel 
Standar Model + Higgs + 3 neutrino kanan secara eksak dari eigenvalue diskrit $L_I$. 
Bab XIV akan menunjukkan bahwa setiap partikel bukanlah entitas terpisah yang 
memerlukan parameter massa ad-hoc, melainkan eigenmode berbeda dari graf RJI--$N$. 
Bab XIV juga akan mengungkap mengapa konstanta struktur halus $\alpha^{-1} \approx 137$ 
muncul sebagai indeks spektral elektron, bukan angka mistik.