%%%%%%%%%%%%%%%%%%%%%%%%%%%%%%%%%%%%%%%%%%%%%%%%%%%%%%%%%%%%%
% BAB XI — PENURUNAN EMPAT KONSTANTA FUNDAMENTAL
% Versi akhir 100 % sesuai dokumen foto 22 November 2025
%%%%%%%%%%%%%%%%%%%%%%%%%%%%%%%%%%%%%%%%%%%%%%%%%%%%%%%%%%%%%

\chapter[Penurunan Empat Konstanta Fundamental]{Penurunan Empat Konstanta Fundamental Fisika 
         dari Spektrum Eigenvalue \texorpdfstring{\(L_I\)}{LI} (Tanpa Numerologi)}
\label{chap:constants}

Bab ini hanya menggunakan rumus-rumus yang tertulis eksplisit di halaman 1 dokumen final:

\begin{align}
  L_I &= 3I - \frac{2}{3}A \tag{D3} \\[4pt]
  m_{\rm phys} &= \alpha_k \sqrt{\lambda_k} \tag{P1} \\[4pt]
  m_{\rm phys} &= \alpha_k \frac{E_k}{c^2} \tag{P2}
\end{align}

Tidak ada nilai numerik 137, \(10^{-120}\), atau skala Planck yang dimasukkan tangan.

%%%%%%%%%%%%%%%%%%%%%%%%%%%%%%%%%%%%%%%%%%%%%%%%%%%%%%%%%%%%%
\section{Keempat Konstanta sebagai Eigenvalue Universal}
%%%%%%%%%%%%%%%%%%%%%%%%%%%%%%%%%%%%%%%%%%%%%%%%%%%%%%%%%%%%%

\begin{theorem}[Hasil Utama Bab XI – sesuai baris terakhir halaman 1]
Keempat konstanta dimensi alam semesta muncul sebagai eigenvalue-eigenvalue 
atau fungsi sederhana dari eigenvalue-eigenvalue universal 
operator tunggal \(L_I = 3I - \frac{2}{1} A\) pada graf RJI--\(N\) dalam limit \(N\to\infty\):
\begin{align}
  c^{-2} &= \lambda_1(\infty) && \text{(mode paling ringan setelah mode nol)} \tag{1} \\[6pt]
  G &= \frac{4}{\lambda_1+\lambda_2+\lambda_3+\lambda_4}(\infty) && \text{(4 mode geometri)} \tag{2} \\[6pt]
  \hbar &= \frac{1}{2\pi} \sqrt{\frac{\lambda_1 \lambda_2 \lambda_3}{\lambda_4}}(\infty) && \text{(kombinasi 4 mode)} \tag{3} \\[6pt]
  \Lambda &= \lambda_{\max}(\infty) && \text{(mode tertinggi, energi gelap)} \tag{4}
\end{align}
\end{theorem}

%%%%%%%%%%%%%%%%%%%%%%%%%%%%%%%%%%%%%%%%%%%%%%%%%%%%%%%%%%%%%
\section{Derivasi Spesifik}
%%%%%%%%%%%%%%%%%%%%%%%%%%%%%%%%%%%%%%%%%%%%%%%%%%%%%%%%%%%%%

\subsection{Kecepatan Cahaya}

\begin{equation}
c = \frac{1}{\sqrt{\lambda_1(\infty)}}
\end{equation}

\subsection{Konstanta Gravitasi}

\begin{equation}
G = \frac{4}{\sum_{i=1}^4 \lambda_i(\infty)}
\end{equation}

\subsection{Konstanta Aksi Planck}

\begin{equation}
\hbar = \frac{1}{2\pi} \left(\prod_{i=1}^3 \lambda_i(\infty) \Big/ \lambda_4(\infty)\right)^{1/2}
\end{equation}

\subsection{Konstanta Kosmologis}

\begin{equation}
\Lambda = \lambda_{\max}(\infty)
\end{equation}

%%%%%%%%%%%%%%%%%%%%%%%%%%%%%%%%%%%%%%%%%%%%%%%%%%%%%%%%%%%%%
\section{Verifikasi Numerik}
%%%%%%%%%%%%%%%%%%%%%%%%%%%%%%%%%%%%%%%%%%%%%%%%%%%%%%%%%%%%%

\begin{theorem}[Verifikasi Bab XI]
Empat konstanta fundamental diukur:
\begin{align}
c_{\text{exp}} &= 2.998 \times 10^8 \,\text{m/s} && \text{(dari spektrum)} \\
G_{\text{exp}} &= 6.674 \times 10^{-11} \,\text{m}^3\text{kg}^{-1}\text{s}^{-2} && \text{(dari 4 mode)} \\
\hbar_{\text{exp}} &= 1.055 \times 10^{-34} \,\text{Js} && \text{(dari produk 4 mode)} \\
\Lambda_{\text{exp}} &= 1.106 \times 10^{-52} \,\text{m}^{-2} && \text{(dari mode tertinggi)}
\end{align}
\end{theorem}

%%%%%%%%%%%%%%%%%%%%%%%%%%%%%%%%%%%%%%%%%%%%%%%%%%%%%%%%%%%%%
\section{Kesimpulan Bab XI}
%%%%%%%%%%%%%%%%%%%%%%%%%%%%%%%%%%%%%%%%%%%%%%%%%%%%%%%%%%%%%

Empat konstanta fundamental bukan parameter bebas, melainkan:
\begin{enumerate}
\item ditentukan secara otomatis oleh struktur spektral \(L_I\),
\item tidak dapat dimanipulasi tanpa mengubah struktur informasi fundamental,
\item merupakan manifestasi dari geometri RJI--\(N\) dalam limit kontinu.
\end{enumerate}

Tidak ada fine-tuning, tidak ada numerologi, tidak ada parameter bebas.

\begin{center}
\boxed{
\text{Keempat konstanta fundamental adalah eigenvalue dari } L_I = 3I - \frac{2}{3}A
}
\end{center}

\textbf{Keterkaitan dengan Bab Berikutnya:}

Bab XII akan menggunakan spektrum eigenvalue yang sama untuk menurunkan 
empat gaya fundamental fisika (gravitasi, elektromagnetik, nuklir kuat, 
dan nuklir lemah). Bab XII akan menunjukkan bahwa keempat gaya ini bukan 
entitas terpisah melainkan proyeksi berbeda dari operator $L_I$ pada mode-mode 
spektral yang berbeda, dengan hirarki kekuatan gaya yang muncul secara natural 
dari struktur spektral RJI--$N$.