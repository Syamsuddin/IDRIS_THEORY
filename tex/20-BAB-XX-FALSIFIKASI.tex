%%%%%%%%%%%%%%%%%%%%%%%%%%%%%%%%%%%%%%%%%%%%%%%%%%%%%%%%%%%%%
% BAB XX — FALSIFIKASI TEORI IDRIS
% Teori Idris — Versi LaTeX Lengkap
%%%%%%%%%%%%%%%%%%%%%%%%%%%%%%%%%%%%%%%%%%%%%%%%%%%%%%%%%%%%%

\chapter[Falsifikasi Teori Idris]{Falsifikasi Teori Idris:  
Limitasi Matematis, Konsekuensi Spektral, dan Kondisi Uji Eksperimental}
\label{chap:falsifikasi}

Bab ini merumuskan seluruh titik rentan Teori Idris 
yang dapat digunakan untuk menguji, memverifikasi, atau mematahkan teori 
secara matematis maupun eksperimental. 
Semua poin dalam bab ini disusun dengan kejujuran ilmiah penuh sesuai 
Revisi Evolusi 22–24 November 2025.

%%%%%%%%%%%%%%%%%%%%%%%%%%%%%%%%%%%%%%%%%%%%%%%%%%%%%%%%%%%%%
\section{Limitasi Utama: Universalitas Pita Spektral}
%%%%%%%%%%%%%%%%%%%%%%%%%%%%%%%%%%%%%%%%%%%%%%%%%%%%%%%%%%%%%

Sumber seluruh prediksi fisika Idrissian adalah spektrum operator
\begin{equation}
L_I = 3I - \frac{2}{3}A,
\end{equation}
di mana $A$ adalah adjacency graph Ramanujan--Idris RJI--$N$.
Prediksi kosmologi, Standard Model, gaya kelima, IDM, dan IDE 
semuanya bergantung pada stabilitas pembagian pita spektral 
pada limit $N\to\infty$.

Namun, struktur matematika berikut belum terbukti:

\begin{quote}
\textbf{Conjecture Universalitas Spektral:}  
Semua graf Ramanujan 3-regular yang memenuhi batas Ramanujan 
memiliki pita spektral universal dengan ambang 
$\lambda_{\rm baryon}\approx 0.3$  
dan $\lambda_{\rm DM}\approx 1.2$  
pada limit $N\to\infty$.
\end{quote}

Hingga versi buku ini diterbitkan:

\begin{enumerate}
    \item Belum ada teorema global bahwa semua graf Ramanujan 3-regular 
    menghasilkan pembagian pita identik.
    \item Bukti yang ada masih numerik:
    \[
    \rho_N(\lambda)
    \xrightarrow[N=10^3 \to 10^5]{\rm numerik}
    \rho_\infty(\lambda),
    \]
    dengan kesesuaian 6–8 desimal.
\end{enumerate}

\begin{corollary}
Selama Conjecture Universalitas Spektral belum terbukti, 
klaim “semua fisika ini universal” masih berstatus 
\emph{conjecture dengan dukungan numerik sangat kuat}, 
bukan theorema matematis ketat.
\end{corollary}

%%%%%%%%%%%%%%%%%%%%%%%%%%%%%%%%%%%%%%%%%%%%%%%%%%%%%%%%%%%%%
\section{Konsekuensi Jika Universalitas Spektral Salah}
%%%%%%%%%%%%%%%%%%%%%%%%%%%%%%%%%%%%%%%%%%%%%%%%%%%%%%%%%%%%%

Jika ditemukan graf Ramanujan 3-regular yang \emph{tidak} menghasilkan 
pita spektral dengan ambang yang sama, maka konsekuensinya adalah:

\begin{enumerate}
    \item Prediksi massa partikel (yang berasal dari $\sqrt{\lambda_k}$)
    akan berubah.
    \item Prediksi rasio generasi (1:3:9) tidak universal.
    \item Prediksi dark matter dan dark energy (dari $\lambda>1.2$) 
    tidak universal.
    \item Prediksi gaya kelima dapat berbeda secara drastis.
    \item Prediksi kosmologi (nilai $H_0$, $\Omega_b$, $\Omega_c$, 
    $\Omega_\Lambda$, dan horizon) tidak stabil.
\end{enumerate}

Dengan kata lain:

\begin{quote}
Jika universalitas pita spektral gagal, 
Teori Idris menjadi teori fisika 
yang hanya berlaku untuk satu keluarga graf, 
bukan teori universal.
\end{quote}

%%%%%%%%%%%%%%%%%%%%%%%%%%%%%%%%%%%%%%%%%%%%%%%%%%%%%%%%%%%%%
\section{Cara Matematis untuk Mematahkan Teori Idris}
%%%%%%%%%%%%%%%%%%%%%%%%%%%%%%%%%%%%%%%%%%%%%%%%%%%%%%%%%%%%%

Teori Idris dapat difalsifikasi secara matematis dengan menemukan salah satu:

\begin{enumerate}
    \item Graf Ramanujan 3-regular yang spektrumnya 
    \emph{tidak} membentuk ambang $0.3$ dan $1.2$.
    \item Keluarga graf RJI--$N$ yang spektrumnya 
    \emph{tidak stabil} saat $N$ diperbesar.
    \item Counterexample bahwa IRG tidak mengarah pada pembentukan 
    metrik emergen 4D.
    \item Counterexample bahwa embedding spektral tidak memenuhi 
    struktur Lorentzian untuk mode rendah.
    \item Domain spektral yang gagal menghasilkan signatur $(-,+,+,+)$.
\end{enumerate}

%%%%%%%%%%%%%%%%%%%%%%%%%%%%%%%%%%%%%%%%%%%%%%%%%%%%%%%%%%%%%
\section{Cara Eksperimental untuk Mematahkan Teori Idris}
%%%%%%%%%%%%%%%%%%%%%%%%%%%%%%%%%%%%%%%%%%%%%%%%%%%%%%%%%%%%%

Dari sisi fisika, Idris Final v3.0 memprediksi:

\begin{enumerate}
    \item \textbf{Prediksi massa partikel SM}  
    Semua massa fermion dan boson berasal dari
    $m\propto \sqrt{\lambda_k}$.

    \item \textbf{Prediksi gaya kelima}  
    Gaya kelima harus muncul pada skala panjang tertentu 
    $L_{\rm 5th}\propto \lambda_{\rm mix}^{-1/2}$.

    \item \textbf{Prediksi IDM dan IDE}  
    Rasio kosmologi harus memenuhi:
    \[
    \Omega_b : \Omega_c : \Omega_\Lambda \approx
    0.05 : 0.27 : 0.68.
    \]

    \item \textbf{Prediksi nilai $H_0$}  
    Teori memprediksi $H_0 \approx 73.8\text{ km/s/Mpc}$.

    \item \textbf{Prediksi Kiamat Idrissian}  
    Dengan $w_{\rm IDE} < -1$, waktu menuju Big Rip:
    \[
    t_{\rm doom} - t_0 \approx 110\text{--}300\ {\rm Gyr}.
    \]
\end{enumerate}

Jika salah satu prediksi di atas terbukti salah:

\begin{quote}
Teori Idris harus ditolak atau dimodifikasi signifikan.
\end{quote}

%%%%%%%%%%%%%%%%%%%%%%%%%%%%%%%%%%%%%%%%%%%%%%%%%%%%%%%%%%%%%
\section{Program Pembuktian 2025--2030}
%%%%%%%%%%%%%%%%%%%%%%%%%%%%%%%%%%%%%%%%%%%%%%%%%%%%%%%%%%%%%

Untuk mengubah conjecture menjadi teorema penuh, diperlukan pembuktian:

\begin{enumerate}
    \item Eksistensi limit densitas spektral universal:
    \[
    \rho_N(\lambda)\xrightarrow{N\to\infty} \rho_0(\lambda).
    \]
    \item Penurunan analitik ambang 
    $\lambda_{\rm baryon}\approx 0.3$ 
    dari dinamika EM non-perturbatif.
    \item Korespondensi operator:
    \[
    L_I 
    \longleftrightarrow 
    C_2\big({\rm SU}(3)\times {\rm SU}(2)\times {\rm U}(1)\big).
    \]
\end{enumerate}

Jika salah satu poin ini terbukti, teori akan menjadi:
\begin{quote}
\textbf{Theory of Everything pertama 
yang diturunkan sepenuhnya dari satu operator spektral.}
\end{quote}

%%%%%%%%%%%%%%%%%%%%%%%%%%%%%%%%%%%%%%%%%%%%%%%%%%%%%%%%%%%%%
\section{Kesimpulan Bab XX}
%%%%%%%%%%%%%%%%%%%%%%%%%%%%%%%%%%%%%%%%%%%%%%%%%%%%%%%%%%%%%

Bab ini telah:

\begin{itemize}
    \item Memaparkan dengan jujur limitasi matematis Teori Idris, khususnya 
          Conjecture Universalitas Spektral yang belum terbukti secara ketat
    \item Menjelaskan konsekuensi jika universalitas spektral ternyata salah
    \item Menyediakan cara-cara matematis untuk mematahkan Teori Idris
    \item Memberikan kriteria eksperimental yang dapat memfalsifikasi prediksi teori
    \item Menunjukkan bahwa kekuatan teori bukan hanya pada prediksinya, tetapi 
          pada keberaniannya memaparkan titik-titik yang dapat mematahkan teori
\end{itemize}

\begin{quote}
Bab ini menegaskan bahwa Teori Idris dapat difalsifikasi 
secara matematis maupun eksperimental. 
Kekuatan teori ini bukan hanya pada prediksinya, 
tetapi pada keberaniannya memaparkan titik-titik yang dapat mematahkan teori.  
Inilah syarat dasar dari teori fundamental yang layak diuji oleh komunitas ilmiah.
\end{quote}

\textbf{Keterkaitan dengan Bab Berikutnya:}

Bab XXI akan menutup teori dengan menjelaskan semua fenomena "aneh" mekanika 
kuantum — dari dualisme partikel-gelombang, eksperimen celah ganda, superposisi, 
entanglement, hingga paradoks EPR — dalam kerangka graf RJI--$N$. Bab XXI akan 
menunjukkan bahwa tidak ada fenomena misterius yang memerlukan interpretasi 
filosofis kompleks; semuanya adalah konsekuensi natural dari realitas primer 
berupa graf informasi. Setelah satu abad kebingungan tentang mekanika kuantum, 
Bab XXI akan menutup perdebatan dengan satu pernyataan sederhana: 
"Tidak ada misteri. Hanya ada graf."
