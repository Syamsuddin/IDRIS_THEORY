%%%%%%%%%%%%%%%%%%%%%%%%%%%%%%%%%%%%%%%%%%%%%%%%%%%%%%%%%%%%
% BAB VIII — ERA EMERGEN SE (POST–PLANCKIAN) — FINAL VERSION
%%%%%%%%%%%%%%%%%%%%%%%%%%%%%%%%%%%%%%%%%%%%%%%%%%%%%%%%%%%%

\chapter{Era Emergen SE (Post–Planckian)}
\label{chap:SE}

Setelah era Planckian (SP) mencapai stabilitas penuh, sistem memasuki
\emph{era emergen} (SE): tahap pertama di mana mode–mode spektral
berenergi rendah dari operator informasi $L_I$ mulai memperlihatkan
struktur kontinu yang dapat diinterpretasikan sebagai ruang–waktu
empat-dimensi Lorentzian. Pada era ini, informasi diskrit hasil evolusi
Drissian–WHI–SP mencapai bentuk geometri diferensial makroskopik.

%%%%%%%%%%%%%%%%%%%%%%%%%%%%%%%%%%%%%%%%%%%%%%%%%%%%%%%%%%%%
\section{Kondisi Awal Era Emergen}
%%%%%%%%%%%%%%%%%%%%%%%%%%%%%%%%%%%%%%%%%%%%%%%%%%%%%%%%%%%%

Pada akhir era SP terdapat tiga kondisi fundamental:

\begin{enumerate}
    \item \textbf{Operator informasi Hermitian stabil}
    \begin{equation}
        L_I = 3I - \frac{2}{3} A,
        \label{eq:LISPfinal}
    \end{equation}
    dengan $A$ matriks adjacency graf Ramanujan–Idris RJI--$N$.

    \item \textbf{Spektrum real dan non–negatif}
    \begin{equation}
        0 = \lambda_0 < \lambda_1 \le \cdots \le \lambda_{N-1}.
    \end{equation}
    Mode nol $\psi_0$ mempertahankan interpretasi sebagai arah waktu
    (warisan dari fase WHI).

    \item \textbf{Peluruhan eksponensial mode tinggi}
    Mode $\psi_k$ dengan $\lambda_k \gg 1$ telah mengalami redaman cepat:
    \begin{equation}
        \psi_k(t) = \psi_k(0) e^{-\lambda_k t},
    \end{equation}
    sehingga hanya mode rendah yang tetap relevan secara makroskopik.
\end{enumerate}

Era SE adalah fase di mana mode–mode rendah ini pertama kali
menyusun struktur geometri kontinu.

%%%%%%%%%%%%%%%%%%%%%%%%%%%%%%%%%%%%%%%%%%%%%%%%%%%%%%%%%%%%
\section{Proyeksi Mode Rendah dan Koordinat Emergen}
%%%%%%%%%%%%%%%%%%%%%%%%%%%%%%%%%%%%%%%%%%%%%%%%%%%%%%%%%%%%

Era SE dibangun dari subset mode–mode spektral rendah:
\begin{equation}
    \mathcal{S}_{\rm low}
    =
    \left\{
        \psi_k \;\middle|\;
        1 \le k \le K,\; K \ll N
    \right\}.
\end{equation}

Koordinat ruang–waktu awal didefinisikan sebagai embedding spektral:
\begin{equation}
    X^\mu(x)
    =
    \sum_{k=1}^{K}
        c_k^{(\mu)} \psi_k(x),
    \label{eq:embeddingSEfinal}
\end{equation}
di mana $c_k^{(\mu)}$ adalah koefisien yang diperoleh dari struktur spektral
RJI--$N$ dan basis ortonormal ruang Hilbert informasi (Bab VIII).

Persamaan (\ref{eq:embeddingSEfinal}) mendefinisikan
koordinat makroskopik pertama dari manifold emergen.

%%%%%%%%%%%%%%%%%%%%%%%%%%%%%%%%%%%%%%%%%%%%%%%%%%%%%%%%%%%%
\section{Metrik Spektral Emergen}
%%%%%%%%%%%%%%%%%%%%%%%%%%%%%%%%%%%%%%%%%%%%%%%%%%%%%%%%%%%%

Metrik emergen didefinisikan melalui proyeksi spektral mode rendah:
\begin{equation}
    g_{\mu\nu}(x)
    =
    \sum_{k=1}^{K}
        \lambda_k^{-1}
        (\partial_\mu \psi_k(x))
        (\partial_\nu \psi_k(x)).
    \label{eq:gSEfinal}
\end{equation}

Metrik ini memiliki sifat:

\begin{itemize}
    \item \textbf{Lorentzian}  
    karena arah waktu telah ditentukan oleh deformasi WHI (Bab IV).

    \item \textbf{Dimensi efektif empat}  
    dari hasil dimensi spektral graf RJI--$N$ (Bab II).

    \item \textbf{Kontinu pada limit $N\to\infty$}  
    dengan $K$ tetap atau tumbuh sub-linear.

    \item \textbf{Stabil pada skala makro}  
    karena mode tinggi telah teredam pada era SP.
\end{itemize}

%%%%%%%%%%%%%%%%%%%%%%%%%%%%%%%%%%%%%%%%%%%%%%%%%%%%%%%%%%%%
\section{Dinamika Mode Rendah pada Era SE}
%%%%%%%%%%%%%%%%%%%%%%%%%%%%%%%%%%%%%%%%%%%%%%%%%%%%%%%%%%%%

Mode–mode rendah berevolusi lambat:
\begin{equation}
    \dot{\psi}_k = -\lambda_k \psi_k,
    \qquad
    \lambda_k \ll 1,\; 1\le k\le K.
\end{equation}

Sehingga pada skala makrokosmik:
\begin{equation}
    \psi_k(t) \approx \text{konstan relatif},
\end{equation}
dan geometri yang dibentuk oleh mode–mode ini bersifat stabil.

Tidak ada interpretasi kosmologis (energi gelap, dark matter,
atau spektrum primordial) yang diberikan pada tahap ini.
Semua interpretasi tersebut baru muncul pada Bab XVII–XX.

%%%%%%%%%%%%%%%%%%%%%%%%%%%%%%%%%%%%%%%%%%%%%%%%%%%%%%%%%%%%
\section{Kondisi Konsistensi Menuju GR}
%%%%%%%%%%%%%%%%%%%%%%%%%%%%%%%%%%%%%%%%%%%%%%%%%%%%%%%%%%%%

Agar metrik emergen (\ref{eq:gSEfinal}) mendekati metrik GR kontinu,
syarat berikut harus dipenuhi:

\begin{enumerate}
    \item \textbf{Limit kontinuum}  
    \[
        N \to \infty \quad \text{dengan } K \text{ tetap}.
    \]

    \item \textbf{Stabilitas spektral}  
    \[
        \lambda_k^{-1}
        \text{ halus terhadap } k.
    \]

    \item \textbf{Koherensi dan ortogonalitas}  
    \[
        \langle \psi_j, \psi_k \rangle_I = \delta_{jk}
        \quad \text{untuk } j,k\le K.
    \]
\end{enumerate}

Derivasi lengkap bahwa limit kontinuum 
menghasilkan \emph{persamaan Einstein emergent}
diberikan pada Bab X–XI dan Lampiran D.

%%%%%%%%%%%%%%%%%%%%%%%%%%%%%%%%%%%%%%%%%%%%%%%%%%%%%%%%%%%%
\section{Interpretasi Fisik Era SE}
%%%%%%%%%%%%%%%%%%%%%%%%%%%%%%%%%%%%%%%%%%%%%%%%%%%%%%%%%%%%

Era SE merupakan tahap di mana:

\begin{itemize}
    \item ruang–waktu empat-dimensi pertama kali muncul sebagai entitas kontinu,  
    \item koordinat makroskopik terbentuk melalui embedding spektral,  
    \item mode–mode rendah $L_I$ menjadi pembawa struktur geometris,  
    \item dinamika informasi per energi rendah menjadi dinamika gravitasi efektif,  
    \item namun entitas fisika (materi, medan, partikel) belum muncul.
\end{itemize}

Era SE adalah \emph{pra-geometri fisika}:  
ruang–waktu sudah ada, tetapi fisika dalam ruang–waktu belum terbentuk.

%%%%%%%%%%%%%%%%%%%%%%%%%%%%%%%%%%%%%%%%%%%%%%%%%%%%%%%%%%%%
\section{Kesimpulan Bab VII}
%%%%%%%%%%%%%%%%%%%%%%%%%%%%%%%%%%%%%%%%%%%%%%%%%%%%%%%%%%%%

Bab ini mendeskripsikan struktur matematis lengkap dari era SE,
tahap transisi dari struktur informasi ke geometri kontinu.
\section{Kesimpulan Bab VIII}

Bab ini telah:

\begin{itemize}
    \item Menjelaskan secara lengkap Era Emergen (SE) sebagai era dimana geometri 
          ruang-waktu muncul secara penuh
    \item Menunjukkan bahwa proyeksi mode rendah dari operator $L_I$ menentukan 
          metrik emergen yang stabil dan Lorentzian
    \item Membuktikan bahwa arah waktu telah dipastikan oleh fase WHI sebelumnya
    \item Menunjukkan bahwa geometri yang terbentuk memiliki tanda $(-,+,+,+)$ 
          yang stabil
    \item Menegaskan bahwa syarat menuju Relativitas Umum telah terpenuhi pada 
          era SE
\end{itemize}

Pada era SE, semua fondasi untuk fisika klasik dan kuantum telah terbentuk, 
namun GR belum sepenuhnya muncul dan memerlukan kerangka matematis tambahan.

\textbf{Keterkaitan dengan Bab Berikutnya:}

Bab IX akan memperkenalkan Ruang Hilbert Informasi (IRG) sebagai kerangka 
matematis formal untuk seluruh mode $\psi_k$ dalam Teori Idris. IRG akan 
menjadi struktur fundamental yang menyatukan seluruh aspek matematis teori, 
dari operator spektral hingga dinamika quantum, dan akan menjadi fondasi 
untuk derivasi Relativitas Umum dan teori medan kuantum pada bab-bab selanjutnya.

%%%%%%%%%%%%%%%%%%%%%%%%%%%%%%%%%%%%%%%%%%%%%%%%%%%%%%%%%%%%
% END OF BAB VII — FINAL VERSION
%%%%%%%%%%%%%%%%%%%%%%%%%%%%%%%%%%%%%%%%%%%%%%%%%%%%%%%%%%%%