% Teori Idris — Bab I–IX (Revisi Konsistensi) % Dokumen ini akan diisi bertahap sesuai instruksi

% === Placeholder Struktur === % Bab I — Supremasi Prinsip Informasi (SPI) % Bab II — Graf Ramanujan–Idris (RJI) dan Stabilitas Spektral % Bab III — Aksi Drissian (SD) % Bab IV — White Hole Informasi (WHI) % Bab V — Transisi SD → WHI → SP % Bab VI — Era Planckian (SP) % Bab VII — Era Emergen (SE) % Bab VIII — Ruang Hilbert Informasi % Bab IX — Informational Renormalization Group

%%%%%%%%%%%%%%%%%%%%%%%%%%%%%%%%%%%%%%%%%%%%%%%%%%%%%%%%%%%% % BAB IX — INFORMATIONAL RENORMALIZATION GROUP (IRG) %%%%%%%%%%%%%%%%%%%%%%%%%%%%%%%%%%%%%%%%%%%%%%%%%%%%%%%%%%%%

\chapter{Informational Renormalization Group (IRG)} \label{chap:IRG}

Informational Renormalization Group (IRG) adalah mekanisme skala yang muncul secara alamiah dalam Teori Idris dan menghubungkan dinamika informasi pada graf Ramanujan--Idris (RJI--$N$) dengan limit kontinuum ketika $N\to\infty$. IRG tidak diperkenalkan sebagai struktur baru, melainkan merupakan konsekuensi langsung dari spektrum operator informasi: \begin{equation} L_I = 3I - \frac{2}{3}A. \end{equation}

%%%%%%%%%%%%%%%%%%%%%%%%%%%%%%%%%%%%%%%%%%%%%%%%%%%%%%%%%%%%
\section{Motivasi IRG}
%%%%%%%%%%%%%%%%%%%%%%%%%%%%%%%%%%%%%%%%%%%%%%%%%%%%%%%%%%%%

Tujuan IRG adalah memahami bagaimana spektrum $L_I$ berubah ketika ukuran graf bertambah. Transformasi skala ini tidak mengubah derajat simpul (tetap 3), melainkan memodifikasi densitas spektral.

%%%%%%%%%%%%%%%%%%%%%%%%%%%%%%%%%%%%%%%%%%%%%%%%%%%%%%%%%%%%
\section{Operator Skala}
%%%%%%%%%%%%%%%%%%%%%%%%%%%%%%%%%%%%%%%%%%%%%%%%%%%%%%%%%%%%

IRG didefinisikan melalui operator: \begin{equation} \mathcal{R}_b : L_I(N) \mapsto L_I(bN), \end{equation} untuk faktor skala $b>1$.

%%%%%%%%%%%%%%%%%%%%%%%%%%%%%%%%%%%%%%%%%%%%%%%%%%%%%%%%%%%%
\section{Persamaan Aliran IRG}
%%%%%%%%%%%%%%%%%%%%%%%%%%%%%%%%%%%%%%%%%%%%%%%%%%%%%%%%%%%%

Aliran IRG ditulis dalam bentuk umum: \begin{equation} \mu \frac{d}{d\mu} L_I(\mu) = \beta[L_I(\mu)], \end{equation} dengan fungsi beta akan diturunkan pada Bab XXI dan Bab XXV.

%%%%%%%%%%%%%%%%%%%%%%%%%%%%%%%%%%%%%%%%%%%%%%%%%%%%%%%%%%%%
\section{Aliran Eigenvalue}
%%%%%%%%%%%%%%%%%%%%%%%%%%%%%%%%%%%%%%%%%%%%%%%%%%%%%%%%%%%%

\begin{equation} \lambda_k(N) \mapsto \lambda_k(bN) = \lambda_k(N) + \delta\lambda_k, \end{equation} dengan koreksi orde: \begin{equation} |\delta\lambda_k| = \mathcal{O}(N^{-1/2}). \end{equation} Mode rendah stabil, sedangkan mode tinggi membentuk densitas kontinu.

%%%%%%%%%%%%%%%%%%%%%%%%%%%%%%%%%%%%%%%%%%%%%%%%%%%%%%%%%%%%
\section{Limit Kontinuum}
%%%%%%%%%%%%%%%%%%%%%%%%%%%%%%%%%%%%%%%%%%%%%%%%%%%%%%%%%%%%

\begin{equation} \rho(\lambda) = \lim_{N\to\infty} \frac{1}{N} \sum_{k=0}^{N-1} \delta(\lambda - \lambda_k). \end{equation}

Metrik kontinu: \begin{equation} g_{\mu\nu}(x) = \int \rho(\lambda), \lambda^{-1} (\partial_{\mu} \psi_\lambda)(\partial_{\nu} \psi_\lambda) d\lambda. \end{equation}

%%%%%%%%%%%%%%%%%%%%%%%%%%%%%%%%%%%%%%%%%%%%%%%%%%%%%%%%%%%%
\section{Peran IRG}
%%%%%%%%%%%%%%%%%%%%%%%%%%%%%%%%%%%%%%%%%%%%%%%%%%%%%%%%%%%%

IRG menjadi jembatan matematis dari SD $\to$ WHI $\to$ SP $\to$ SE $\to$ GR emergen.

%%%%%%%%%%%%%%%%%%%%%%%%%%%%%%%%%%%%%%%%%%%%%%%%%%%%%%%%%%%%
\section{Kesimpulan}
%%%%%%%%%%%%%%%%%%%%%%%%%%%%%%%%%%%%%%%%%%%%%%%%%%%%%%%%%%%% IRG mengatur evolusi spektral, menghubungkan diskrit ke kontinu, dan menjadi dasar geometri emergen. (IRG)

% Konten lengkap akan ditambahkan pada update berikutnya.

