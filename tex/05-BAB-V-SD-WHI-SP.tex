%%%%%%%%%%%%%%%%%%%%%%%%%%%%%%%%%%%%%%%%%%%%%%%%%%%%%%%%%%%%%%
% BAB V — TRANSISI INFORMASIONAL SD → WHI → SP (Final Version)
%%%%%%%%%%%%%%%%%%%%%%%%%%%%%%%%%%%%%%%%%%%%%%%%%%%%%%%%%%%%%%

\chapter{Transisi Informasional \texorpdfstring{SD $\rightarrow$ WHI $\rightarrow$ SP}{SD ke WHI ke SP}}
\label{chap:SD-WHI-SP}

Bab ini menjelaskan struktur matematis transisi lintas-era yang 
menghubungkan Aksi Drissian (SD), fase non-Hermitian transien 
(White Hole Informasi / WHI), dan fase Planckian (SP). 
Semua rumusan mengikuti fondasi yang telah ditetapkan pada 
Bab I–IV dan menggunakan notasi konsisten dari Teori Idris.

%%%%%%%%%%%%%%%%%%%%%%%%%%%%%%%%%%%%%%%%%%%%%%%%%%%%%%%%%%%%%%
\section{Aksi Dasar dan Operator Informasi}
%%%%%%%%%%%%%%%%%%%%%%%%%%%%%%%%%%%%%%%%%%%%%%%%%%%%%%%%%%%%%%

Aksi Drissian didefinisikan sebagai:
\begin{equation}
    S_D[I] = \sum_{i,j} A_{ij}\, I_i I_j,
    \label{eq:SD-final}
\end{equation}
dengan $A$ adjacency matrix graf Ramanujan--Idris (RJI--$N$). 
Variasi aksi memberikan:
\begin{equation}
    L_I I = 0, 
    \qquad 
    L_I = 3I - \frac{2}{3}A.
\end{equation}

Spektrum eigen operator $L_I$:
\[
    L_I \psi_k = \lambda_k \psi_k
\]
melahirkan struktur dasar ruang Hilbert informasi 
$\mathcal{H}_I = \mathrm{span}\{\psi_k\}$.

%%%%%%%%%%%%%%%%%%%%%%%%%%%%%%%%%%%%%%%%%%%%%%%%%%%%%%%%%%%%%%
\section{Kondisi Menuju WHI: Generiknya Fluktuasi Antisimetri}
%%%%%%%%%%%%%%%%%%%%%%%%%%%%%%%%%%%%%%%%%%%%%%%%%%%%%%%%%%%%%%

Graf RJI--$N$ ideal bersifat simetris ($A=A^T$), namun pada sistem fisik 
nyata dengan $N$ terbatas, fluktuasi topologis lokal tidak selalu 
mempertahankan kesimetrian sempurna. 

\textbf{Tidak diklaim} bahwa simetri pasti pecah atau terdapat 
batas maksimum densitas informasi.

Sebaliknya, kami menyatakan secara \emph{konservatif dan jujur}:

\begin{quote}
    Pada graf finite-$N$, term antisimetri kecil 
    $K_{\rm skew}$ dengan $K_{\rm skew}^T = -K_{\rm skew}$
    muncul secara generik sebagai akibat ketidaksempurnaan lokal, noise,
    atau dinamika informasi yang tidak sepenuhnya reversibel.
\end{quote}

Ini menghasilkan operator efektif:
\begin{equation}
    A \;\longrightarrow\;
    A + \varepsilon K_{\rm skew},
    \qquad 
    \varepsilon \ll 1.
    \label{eq:A-perturb-final}
\end{equation}

Bab XII menunjukkan bahwa mekanisme ini konsisten dengan SPI 
dan stabil dalam konteks IRG.

%%%%%%%%%%%%%%%%%%%%%%%%%%%%%%%%%%%%%%%%%%%%%%%%%%%%%%%%%%%%%%
\section{Mode Nol Informasi dan Pecahnya Hermiticity}
%%%%%%%%%%%%%%%%%%%%%%%%%%%%%%%%%%%%%%%%%%%%%%%%%%%%%%%%%%%%%%

Pada keadaan simetris,
\[
    L_I \psi_0 = 0.
\]

Dengan perturbasi (\ref{eq:A-perturb-final}), operator menjadi:
\[
    L^{\rm (WHI)}_I
    = 3I - \frac{2}{3}\bigl(A + \varepsilon K_{\rm skew}\bigr).
\]

Analisis gangguan pertama memberikan pergeseran:
\begin{equation}
    \lambda_0^{(\mathrm{WHI})}
    = i\varepsilon 
    \langle \psi_0, K_{\rm skew} \psi_0\rangle,
    \qquad \epsilon = \varepsilon |\langle \psi_0, K_{\rm skew} \psi_0\rangle|.
\end{equation}

Karena $K_{\rm skew}$ antisimetri:
\[
    \lambda_0^{(\mathrm{WHI})} = i\epsilon, \qquad \epsilon>0.
\]

%%%%%%%%%%%%%%%%%%%%%%%%%%%%%%%%%%%%%%%%%%%%%%%%%%%%%%%%%%%%%%
\section{Definisi Formal White Hole Informasi}
%%%%%%%%%%%%%%%%%%%%%%%%%%%%%%%%%%%%%%%%%%%%%%%%%%%%%%%%%%%%%%

\begin{definition}[White Hole Informasi (WHI)]
Fase WHI adalah fase transien ketika operator informasi mengalami
perturbasi antisimetri kecil sehingga mode nol memperoleh eigenvalue
imajiner:
\[
    \lambda_0^{(\mathrm{WHI})}=i\epsilon.
\]
Definisi ini memberikan \emph{kondisi cukup}, bukan kondisi perlu,
untuk munculnya arah waktu informasional.
\end{definition}

%%%%%%%%%%%%%%%%%%%%%%%%%%%%%%%%%%%%%%%%%%%%%%%%%%%%%%%%%%%%%%
\section{Pemulihan Hermiticity dan Fase Planckian (SP)}
%%%%%%%%%%%%%%%%%%%%%%%%%%%%%%%%%%%%%%%%%%%%%%%%%%%%%%%%%%%%%%

Setelah mode waktu terbentuk, perturbasi antisimetri meluruh:
\[
    \varepsilon(t)\rightarrow 0.
\]

Operator kembali Hermitian:
\[
    L_I^{\rm (SP)} = 3I - \frac{2}{3}A,
\]
dengan satu arah waktu telah terkunci secara intrinsik. 
Fase ini adalah fase Planckian (SP).

%%%%%%%%%%%%%%%%%%%%%%%%%%%%%%%%%%%%%%%%%%%%%%%%%%%%%%%%%%%%%%
\section{Dinamika Informasi Pasca-WHI}
%%%%%%%%%%%%%%%%%%%%%%%%%%%%%%%%%%%%%%%%%%%%%%%%%%%%%%%%%%%%%%

Pada fase SP mode non-nol berevolusi menurut dinamika efektif:
\begin{equation}
    \dot{\psi}_k 
    = - \lambda_k \psi_k,
    \label{eq:psi-evolve-final}
\end{equation}
yang akan diturunkan secara detail pada Bab VII sebagai
kasus khusus dari prinsip IRG.

Perubahan entropi informasional:
\[
    S_I(t) = S_I(0) + \sum_{k>0} \lambda_k |\psi_k(t)|^2.
\]

%%%%%%%%%%%%%%%%%%%%%%%%%%%%%%%%%%%%%%%%%%%%%%%%%%%%%%%%%%%%%%
\section{Konservasi Informasi Lintas-Era}
%%%%%%%%%%%%%%%%%%%%%%%%%%%%%%%%%%%%%%%%%%%%%%%%%%%%%%%%%%%%%%

Aksioma A3 Teori Idris menyatakan:
\begin{equation}
    N_{\rm Drissian} 
    = N_{\rm WHI}
    = N_{\rm SP}
    = N_{\rm SE},
    \label{eq:info-conserve-final}
\end{equation}
yang berarti WHI mengubah \emph{representasi} informasi, 
bukan kuantitasnya.

%%%%%%%%%%%%%%%%%%%%%%%%%%%%%%%%%%%%%%%%%%%%%%%%%%%%%%%%%%%%%%
\section{Kesimpulan Bab V}
%%%%%%%%%%%%%%%%%%%%%%%%%%%%%%%%%%%%%%%%%%%%%%%%%%%%%%%%%%%%%%

Bab ini telah:

\begin{itemize}
    \item Menjelaskan transisi lengkap SD → WHI → SP sebagai rantai evolusi 
          informasional yang melahirkan struktur fisika
    \item Menunjukkan bagaimana fase Wheeler-Hawking Informasional (WHI) 
          menjadi jembatan kritis antara era pra-geometrik dan geometrik
    \item Membuktikan munculnya arah waktu informasional melalui WHI
    \item Menunjukkan munculnya tanda Lorentzian $(-,+,+,+)$ secara natural
    \item Memperkenalkan operator Hermitian stabil untuk era Planckian
    \item Meletakkan fondasi untuk struktur ruang Hilbert fisika
\end{itemize}

Transisi ini adalah jembatan matematis antara aksi informasi murni
dan geometri ruang-waktu fisik yang muncul pada era SE.

\textbf{Keterkaitan dengan Bab Berikutnya:}

Bab VI akan fokus pada era Planckian (SP) secara lebih mendalam, menjelaskan 
bagaimana aksi Planckian bekerja pada skala energi Planck dan mengapa era ini 
menjadi titik transisi kritis menuju era emergen. Bab VI akan menunjukkan 
bahwa SP adalah jembatan antara fisika pra-geometrik WHI dan fisika geometrik SE.