%%%%%%%%%%%%%%%%%%%%%%%%%%%%%%%%%%%%%%%%%%%%%%%%%%%%%%%%%%%%%
% BAB XIX — MULTIVERSE IDRISSIAN
% Teori Idris — Versi LaTeX Lengkap
%%%%%%%%%%%%%%%%%%%%%%%%%%%%%%%%%%%%%%%%%%%%%%%%%%%%%%%%%%%%%

\chapter[Multiverse Idrissian]{Multiverse Idrissian:
Domain Spektral, Ruang–Waktu Emergen, dan Relativitas Antar-Semesta}
\label{chap:multiverse-idris}

Multiverse Idrissian muncul sebagai konsekuensi matematis langsung
dari struktur spektral operator informasi
\begin{equation}
L_I = 3I - \frac{2}{3}A,
\label{eq:LI-multiverse}
\end{equation}
yang bekerja pada Ruang Hilbert Informasi $\mathcal{H}_I$.
Tidak ada parameter tambahan, tidak ada medan baru, dan tidak ada postulat kosmologis.
Multiverse muncul murni karena:
\begin{enumerate}
    \item adanya klaster eigenvalue yang stabil,
    \item invariansi domain spektral di bawah aliran IRG,
    \item kemampuan setiap klaster untuk membangun metrik emergen sendiri.
\end{enumerate}

%%%%%%%%%%%%%%%%%%%%%%%%%%%%%%%%%%%%%%%%%%%%%%%%%%%%%%%%%%%%%
\section{Domain Spektral sebagai Semesta Idrissian}
%%%%%%%%%%%%%%%%%%%%%%%%%%%%%%%%%%%%%%%%%%%%%%%%%%%%%%%%%%%%%

Definisikan himpunan eigenmode dengan nilai eigen pada interval spektral tertentu:
\begin{equation}
\mathcal{D}_\alpha = \{\psi_k : \lambda_k \in I_\alpha\}.
\label{eq:domain-def}
\end{equation}

\begin{definition}[Semesta Idrissian]
Sebuah \emph{semesta Idrissian} adalah domain spektral 
$\mathcal{D}_\alpha$ yang stabil di bawah aliran IRG dan 
memiliki cukup banyak mode rendah untuk membentuk geometri 4D emergen.
\end{definition}

Metrik emergen untuk domain $\alpha$ diberikan oleh embedding spektral:
\begin{equation}
g^{(\alpha)}_{\mu\nu}(x)
=
\sum_{\psi_k \in \mathcal{D}_\alpha}
\lambda_k^{-1}
\partial_\mu\psi_k(x)\,
\partial_\nu\psi_k(x).
\label{eq:metric-alpha}
\end{equation}

Jika dua domain spektral $\mathcal{D}_\alpha$ dan $\mathcal{D}_\beta$ tidak beririsan,
maka metrik $g^{(\alpha)}_{\mu\nu}$ dan $g^{(\beta)}_{\mu\nu}$ menggambarkan
dua semesta yang sepenuhnya independen.

%%%%%%%%%%%%%%%%%%%%%%%%%%%%%%%%%%%%%%%%%%%%%%%%%%%%%%%%%%%%%
\section{Aliran IRG dan Independensi Antar-Semesta}
%%%%%%%%%%%%%%%%%%%%%%%%%%%%%%%%%%%%%%%%%%%%%%%%%%%%%%%%%%%%%

Aliran renormalisasi informasional (IRG) diberikan oleh:
\begin{equation}
\frac{dc_k}{d\tau} = -\lambda_k c_k,
\qquad
c_k(\tau) = c_k(0)e^{-\lambda_k \tau}.
\label{eq:IRG-flow}
\end{equation}

Parameter $\tau$ adalah \emph{waktu informasi global} yang sama untuk seluruh 
$\mathcal{H}_I$. Namun, setiap semesta $\alpha$ mengubah $\tau$ menjadi 
\emph{waktu proper} lokal $t_\alpha$ melalui metriknya sendiri:
\begin{equation}
dt_\alpha^2 = -g^{(\alpha)}_{\mu\nu} dx^\mu dx^\nu.
\label{eq:proper-time-alpha}
\end{equation}

Karena domain-domain spektral yang berbeda memiliki struktur $\lambda_k$
yang berbeda, maka pemetaan
\[
t_\alpha = f_\alpha(\tau)
\]
tidak identik untuk $\alpha$ berbeda.

Inilah dasar munculnya perbedaan skala waktu antar-sekesta.

%%%%%%%%%%%%%%%%%%%%%%%%%%%%%%%%%%%%%%%%%%%%%%%%%%%%%%%%%%%%%
\section{Skala Waktu Antar-Semesta}
%%%%%%%%%%%%%%%%%%%%%%%%%%%%%%%%%%%%%%%%%%%%%%%%%%%%%%%%%%%%%

Waktu fisik dalam semesta $\alpha$ ditetapkan oleh proses lokal
yang bergantung pada massa:
\begin{equation}
m^{(\alpha)} \sim \sqrt{\lambda^{(\alpha)}},
\label{eq:mass-lambda}
\end{equation}
sehingga skala waktu fundamental:
\begin{equation}
T_\alpha \sim \frac{1}{m^{(\alpha)}} 
\sim \frac{1}{\sqrt{\lambda^{(\alpha)}}}.
\label{eq:T-alpha}
\end{equation}

Dua semesta $\alpha$ dan $\beta$ akan memiliki rasio skala waktu:
\begin{equation}
\frac{T_\alpha}{T_\beta}
=
\sqrt{
\frac{\lambda_\beta}{\lambda_\alpha}
}.
\label{eq:ratio-time}
\end{equation}

\begin{proposition}[Relativitas Antar-Semesta]
Jika spektrum domain $\mathcal{D}_\alpha$ dan $\mathcal{D}_\beta$ berbeda 
dengan faktor
\[
\frac{\lambda_\beta}{\lambda_\alpha} \approx (365)^2,
\]
maka
\[
T_\alpha \approx 365\, T_\beta.
\]
Dalam hal ini, satu interval informasi $\Delta\tau$ yang sama dapat 
ditafsirkan sebagai 1 hari di semesta $\alpha$ dan 1 tahun di semesta $\beta$.
\end{proposition}

Meskipun demikian, tidak ada mekanisme interaksi antar-domain,
karena:
\begin{equation}
\langle \psi^{(\alpha)}, L_I \psi^{(\beta)} \rangle = 0,
\qquad
\mathcal{D}_\alpha \cap \mathcal{D}_\beta = \emptyset.
\label{eq:orthogonality}
\end{equation}

Waktu mereka hanya dapat dibandingkan oleh pengamat “di luar teori”,
yakni pemodel spektral, bukan oleh entitas fisik dalam semesta itu sendiri.

%%%%%%%%%%%%%%%%%%%%%%%%%%%%%%%%%%%%%%%%%%%%%%%%%%%%%%%%%%%%%
\section{Struktur Multiverse Idrissian}
%%%%%%%%%%%%%%%%%%%%%%%%%%%%%%%%%%%%%%%%%%%%%%%%%%%%%%%%%%%%%

Dari konstruksi spektral di atas, Multiverse Idrissian memiliki sifat:

\begin{enumerate}
    \item \textbf{Satu ruang Hilbert, banyak domain:}
    \[
    \mathcal{H}_I = 
    \bigoplus_\alpha \mathcal{D}_\alpha.
    \]
    
    \item \textbf{Satu parameter evolusi global $\tau$, 
    tetapi banyak waktu lokal $t_\alpha$.}

    \item \textbf{Geometri setiap semesta independen:}
    \[
    g^{(\alpha)}_{\mu\nu} \neq g^{(\beta)}_{\mu\nu}.
    \]

    \item \textbf{Tidak ada interaksi antar semesta:}
    mode-mode di domain berbeda ortogonal.

    \item \textbf{Setiap domain membentuk ruang-waktu 
    hanya jika memiliki setidaknya empat mode rendah.}

    \item \textbf{Alam kita = domain spektral rendah 
    ($\lambda < 0.3$).}
\end{enumerate}

%%%%%%%%%%%%%%%%%%%%%%%%%%%%%%%%%%%%%%%%%%%%%%%%%%%%%%%%%%%%%
\section{IDE, IDM, dan Proyeksi Antar-Semesta}
%%%%%%%%%%%%%%%%%%%%%%%%%%%%%%%%%%%%%%%%%%%%%%%%%%%%%%%%%%%%%

Mode dengan $\lambda > 1.2$ membentuk energi gelap 
dalam semesta kita:
\begin{equation}
\rho_{\rm IDE}
=
\sum_{\lambda_k > 1.2} \frac{1}{2}\lambda_k |c_k|^2.
\end{equation}

Namun dalam domain asalnya sendiri, mode-mode ini
membangun geometri penuh:
\begin{equation}
g^{({\rm high})}_{\mu\nu}(x)
=
\sum_{\lambda>1.2}
\lambda^{-1}
\partial_\mu\psi\,\partial_\nu\psi.
\end{equation}

Dengan demikian:
\begin{quote}
IDE dalam semesta kita adalah proyeksi jauh 
dari domain spektral tinggi yang membentuk 
“semesta lain” dalam multiverse.
\end{quote}

%%%%%%%%%%%%%%%%%%%%%%%%%%%%%%%%%%%%%%%%%%%%%%%%%%%%%%%%%%%%%
\section{Syarat Eksistensi Semesta Baru}
%%%%%%%%%%%%%%%%%%%%%%%%%%%%%%%%%%%%%%%%%%%%%%%%%%%%%%%%%%%%%

Sebuah domain spektral $I_\alpha$ membentuk semesta baru jika:

\begin{enumerate}
    \item Memiliki empat mode rendah internal:
    \[
    \lambda_{a_1},\ldots,\lambda_{a_4}\in I_\alpha.
    \]
    \item Memiliki densitas spektral kontinu:
    \[
    \rho(\lambda)|_{I_\alpha} > 0.
    \]
    \item Stabil di bawah IRG:
    \[
    c_k(\tau) \neq 0 \text{ untuk } k\in\mathcal{D}_\alpha.
    \]
\end{enumerate}

Jika salah satu syarat ini gagal, 
domain tersebut tidak membentuk ruang-waktu.

%%%%%%%%%%%%%%%%%%%%%%%%%%%%%%%%%%%%%%%%%%%%%%%%%%%%%%%%%%%%%
\section{Perbedaan Konsep Waktu Antar-Semesta}
%%%%%%%%%%%%%%%%%%%%%%%%%%%%%%%%%%%%%%%%%%%%%%%%%%%%%%%%%%%%%

Setiap semesta memiliki pemetaan:
\begin{equation}
t_\alpha = f_\alpha(\tau),
\end{equation}
dengan $f_\alpha$ ditentukan oleh $g^{(\alpha)}$.

Karena bentuk metrik:
\[
ds^2 = -dt_\alpha^2 + a_\alpha^2(t_\alpha)\, d\vec{x}^2,
\]
waktu proper berkembang berbeda untuk setiap domain.

\begin{proposition}[Ketidakseragaman Skala Waktu Multiverse]
Dua semesta dapat memiliki waktu lokal yang berbeda 
hingga faktor besar:
\[
\frac{dt_\alpha}{dt_\beta}
=
\sqrt{
\frac{\lambda_\beta}{\lambda_\alpha}
},
\]
sehingga wajar jika sebuah interval waktu yang sama 
dalam $\tau$ dapat dibaca sebagai 1 hari di semesta A 
dan 1 tahun di semesta B.
\end{proposition}

Hal ini konsisten dengan SPI–PAMI karena:
\begin{itemize}
    \item $\tau$ adalah waktu informasi global yang unik.
    \item $t_\alpha$ adalah waktu fisik lokal yang muncul dari geometri domain.
    \item Tidak ada kontradiksi karena domain-domain tidak berinteraksi.
\end{itemize}

%%%%%%%%%%%%%%%%%%%%%%%%%%%%%%%%%%%%%%%%%%%%%%%%%%%%%%%%%%%%%
\section{Kesimpulan Bab XVIII}
%%%%%%%%%%%%%%%%%%%%%%%%%%%%%%%%%%%%%%%%%%%%%%%%%%%%%%%%%%%%%

Bab ini telah:

\begin{itemize}
    \item Menunjukkan bahwa Multiverse Idrissian adalah konsekuensi langsung 
          dari struktur spektral operator $L_I$
    \item Membuktikan bahwa setiap domain spektral stabil dapat membentuk 
          semesta independen dengan geometri dan skala waktu sendiri
    \item Menjelaskan relativitas antar-semesta dimana waktu informasi global 
          $\tau$ dapat dipetakan ke waktu proper lokal $t_\alpha$ yang berbeda
    \item Menunjukkan bahwa semesta-semesta ini tidak berinteraksi karena 
          ortogonalitas domain spektral
\end{itemize}

\begin{quote}
Multiverse Idrissian adalah konsekuensi matematis murni, bukan spekulasi. 
Setiap klaster eigenvalue stabil membentuk semesta independen dengan 
konstanta fundamental dan skala waktu yang dapat berbeda secara dramatis.
\end{quote}

\textbf{Keterkaitan dengan Bab Berikutnya:}

Bab XIX akan mengeksplorasi dinamika Energi Gelap Idrissian (IDE) secara lebih 
mendalam, menurunkan nilai $w_{\rm IDE}$ (equation of state parameter) dan 
menunjukkan bahwa IDE memiliki karakter phantom ($w < -1$) yang mengarah pada 
prediksi "Big Rip Spektral" di masa depan. Bab XIX akan melengkapi pemahaman 
kita tentang evolusi kosmologis alam semesta dalam kerangka Teori Idris.
