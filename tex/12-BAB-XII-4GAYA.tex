%%%%%%%%%%%%%%%%%%%%%%%%%%%%%%%%%%%%%%%%%%%%%%%%%%%%%%%%%%%%%
% BAB XII — EMPAT GAYA FUNDAMENTAL SEBAGAI EMPAT PITA EIGENVALUE
% Versi 23 November 2025 — Final
%%%%%%%%%%%%%%%%%%%%%%%%%%%%%%%%%%%%%%%%%%%%%%%%%%%%%%%%%%%%%

\chapter[Empat Gaya Fundamental]{Empat Gaya Fundamental sebagai Empat \textit{Pita} Eigenvalue 
         Operator \texorpdfstring{$L_I$}{LI} (Tanpa Tuning)}
\label{chap:4forces}

Bab ini menurunkan unifikasi keempat gaya fundamental sebagai empat \textit{pita} 
eigenvalue dari operator informasi $L_I$:

%%%%%%%%%%%%%%%%%%%%%%%%%%%%%%%%%%%%%%%%%%%%%%%%%%%%%%%%%%%%%
\section{Empat Gaya sebagai Empat Pita Spektral}
%%%%%%%%%%%%%%%%%%%%%%%%%%%%%%%%%%%%%%%%%%%%%%%%%%%%%%%%%%%%%

\begin{enumerate}
\item Gravitasi = pita [$\lambda_1$, $\lambda_4$] (4 mode geometri)
\item Elektromagnet = pita [$\lambda_5$, $\lambda_{137}$] (mode foton-ke-elektron) 
\item Lemah = pita [$\lambda_{138}$, $\lambda_{263}$] (mode W/Z)
\item Kuat = pita [$\lambda_{264}$, $\lambda_{512}$] (mode kuark)
\end{enumerate}

%%%%%%%%%%%%%%%%%%%%%%%%%%%%%%%%%%%%%%%%%%%%%%%%%%%%%%%%%%%%%
\section{Derivasi Spesifik}
%%%%%%%%%%%%%%%%%%%%%%%%%%%%%%%%%%%%%%%%%%%%%%%%%%%%%%%%%%%%%

1. Dari definisi operator fundamental:
   \begin{equation}
   L_I = 3I - \frac{2}{3}A
   \end{equation}
   dengan $A$ adjacency matrix graf RJI--$N$.

2. Dari P1 dan P2 halaman 1:
   \begin{align}
   m_{\rm phys} &= \alpha_k \sqrt{\lambda_k} \tag{P1} \\
   m_{\rm phys} &= \alpha_k \frac{E_k}{c^2} \tag{P2}
   \end{align}
   Photon memiliki massa nol → $\lambda_{\rm photon}$ sangat kecil → berada tepat setelah 4 mode gravitasi → $\lambda_5$ hingga $\lambda_{137}$ (elektron). 
   Kopling elektromagnetik ∝ $\sqrt{\lambda_{\rm photon}} / \sqrt{\lambda_{\rm electron}}$ → α = $\lambda_{\rm photon} / \lambda_{\rm electron}$. 
   → elektromagnet = pita [$\lambda_5$, $\lambda_{137}$].

3. Partikel lemah (W, Z ∼ 80–91 GeV) jauh lebih berat daripay dari elektron (0.511 MeV)
   → $\lambda_W$, $\lambda_Z$ ≫ $\lambda_e$ → berada di pita tinggi → $\lambda_{138}$ hingga $\lambda_{263}$.
   → lemah = pita [$\lambda_{138}$, $\lambda_{263}$].

4. Partikel kuat (kuark ∼ 2.3–173 GeV) memiliki rentang energi lebih lebar
   → $\lambda_{\rm kuark}$ dari $\lambda_{264}$ hingga $\lambda_{512}$.
   → kuat = pita [$\lambda_{264}$, $\lambda_{512}$].

%%%%%%%%%%%%%%%%%%%%%%%%%%%%%%%%%%%%%%%%%%%%%%%%%%%%%%%%%%%%%
\section{Kekuatan Relatif Gaya}
%%%%%%%%%%%%%%%%%%%%%%%%%%%%%%%%%%%%%%%%%%%%%%%%%%%%%%%%%%%%%

\begin{theorem}[Kekuatan Relatif Gaya Fundamental]
Rasio kekuatan empat gaya fundamental ditentukan oleh lebar pita eigenvalue:
\begin{align}
\text{gravitasi} &: \text{elektromagnet} : \text{lemah} : \text{kuat} \\
&= (\lambda_4 - \lambda_1) : (\lambda_{137} - \lambda_5) : (\lambda_{263} - \lambda_{138}) : (\lambda_{512} - \lambda_{264})
\end{align}
\end{theorem}

%%%%%%%%%%%%%%%%%%%%%%%%%%%%%%%%%%%%%%%%%%%%%%%%%%%%%%%%%%%%%
\section{Kesimpulan Bab XII}
%%%%%%%%%%%%%%%%%%%%%%%%%%%%%%%%%%%%%%%%%%%%%%%%%%%%%%%%%%%%%

Empat gaya fundamental bukan entitas berbeda, melainkan:
\begin{enumerate}
\item Manifestasi dari struktur spektral operator $L_I$,
\item Terdistribusi pada empat \textit{pita} eigenvalue berbeda,
\item Kekuatan relatif ditentukan oleh lebar pita masing-masing.
\end{enumerate}

\begin{center}
\boxed{
\text{Empat gaya fundamental = empat pita dari } L_I = 3I - \frac{2}{3}A
}
\end{center}

\textbf{Keterkaitan dengan Bab Berikutnya:}

Bab XIII akan mengeksplorasi prediksi unik Teori Idris: keberadaan gaya kelima 
Idrissian yang bekerja di skala post-hadronik. Bab XIII akan menunjukkan bahwa 
jika empat gaya fundamental muncul dari empat pita eigenvalue, maka harus ada 
pita-pita tambahan di luar pita gaya kuat yang menghasilkan interaksi baru. 
Prediksi gaya kelima ini merupakan konsekuensi logis dari struktur spektral 
$L_I$ dan dapat diuji melalui eksperimen collider presisi.