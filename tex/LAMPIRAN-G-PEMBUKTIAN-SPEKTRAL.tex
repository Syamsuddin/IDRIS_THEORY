%%%%%%%%%%%%%%%%%%%%%%%%%%%%%%%%%%%%%%%%%%%%%%%%%%%%%%%%%%%%%
% LAMPIRAN — METODOLOGI PEMBUKTIAN UNIVERSALITAS SPEKTRAL
% Teori Idris F — Versi LaTeX Lengkap
%%%%%%%%%%%%%%%%%%%%%%%%%%%%%%%%%%%%%%%%%%%%%%%%%%%%%%%%%%%%%

\chapter{Metodologi Pembuktian Universalitas Spektral}
\label{appendix:metodologi-USBC}

Lampiran ini menjelaskan metodologi matematis formal yang dibutuhkan 
untuk membuktikan \emph{Universal Spectral Band Conjecture} (USBC), yaitu:

\begin{quote}
Semua graf Ramanujan 3-regular $G_N$ memiliki densitas spektral 
yang konvergen ke fungsi universal $\rho_\infty(\lambda)$
dengan struktur pita tetap 
(\,$\lambda_{\rm baryon}\approx 0.3$, 
$\lambda_{\rm DM}\approx 1.2$,  
dll) 
pada limit kontinuum $N\to\infty$.
\end{quote}

Pembuktian ini merupakan langkah fundamental menuju penyempurnaan 
Teori Idris menjadi teori yang terbukti secara matematis penuh.

%%%%%%%%%%%%%%%%%%%%%%%%%%%%%%%%%%%%%%%%%%%%%%%%%%%%%%%%%%%%%
\section{Struktur Operator dan Objek Matematis}
%%%%%%%%%%%%%%%%%%%%%%%%%%%%%%%%%%%%%%%%%%%%%%%%%%%%%%%%%%%%%

Operator informasi didefinisikan sebagai:
\begin{equation}
L_I(N) = 3I - \frac{2}{3}A_N,
\label{eq:LI-def}
\end{equation}
dengan $A_N$ adjacency matrix graf Ramanujan--Idris berukuran $N\times N$.

Spektrum eigenvalue:
\begin{equation}
L_I \psi_k = \lambda_k \psi_k,
\qquad 0 \le k \le N-1.
\end{equation}

Distribusi spektral ternormalisasi:
\begin{equation}
\rho_N(\lambda)
=
\frac{1}{N}
\sum_{k=0}^{N-1}
\delta(\lambda - \lambda_k).
\label{eq:rhoN}
\end{equation}

Tujuan pembuktian:
\begin{equation}
\rho_N(\lambda)
\xrightarrow[N\to\infty]{\rm weak}
\rho_\infty(\lambda),
\label{eq:weak-limit}
\end{equation}
dengan \emph{bentuk pita yang universal}.

%%%%%%%%%%%%%%%%%%%%%%%%%%%%%%%%%%%%%%%%%%%%%%%%%%%%%%%%%%%%%
\section{Metode 1: Analisis Resolvent dan Transformasi Stieltjes}
%%%%%%%%%%%%%%%%%%%%%%%%%%%%%%%%%%%%%%%%%%%%%%%%%%%%%%%%%%%%%

Definisikan resolvent operator:
\begin{equation}
R_N(z) = (L_I - zI)^{-1}.
\end{equation}

Trace resolvent:
\begin{equation}
m_N(z) = \frac{1}{N} \mathrm{Tr}(R_N(z)),
\end{equation}
adalah transformasi Stieltjes dari $\rho_N(\lambda)$:
\begin{equation}
m_N(z) 
=
\int \frac{1}{\lambda - z} \, \rho_N(\lambda)\, d\lambda.
\end{equation}

Kunci pembuktian:
\begin{enumerate}
    \item Menunjukkan bahwa $m_N(z)$ konvergen untuk setiap $z\notin\mathbb{R}$.
    \item Menunjukkan batas $m_\infty(z)$ analitik.
    \item Melakukan inversi transformasi Stieltjes untuk mendapatkan 
    $\rho_\infty(\lambda)$.
\end{enumerate}

Jika langkah 1–3 berhasil:
\[
\rho_N \to \rho_\infty.
\]

%%%%%%%%%%%%%%%%%%%%%%%%%%%%%%%%%%%%%%%%%%%%%%%%%%%%%%%%%%%%%
\section{Metode 2: Teknik Local Weak Convergence (Benjamini–Schramm)}
%%%%%%%%%%%%%%%%%%%%%%%%%%%%%%%%%%%%%%%%%%%%%%%%%%%%%%%%%%%%%

Graf Ramanujan 3-regular memenuhi sifat expander kuat:
\[
h(G_N) \ge h_0 > 0.
\]

Dengan syarat ini, secara lokal:
\[
G_N \xrightarrow{\rm local} T_3
\]
dalam arti Benjamini–Schramm, di mana $T_3$ adalah pohon 3-regular tak hingga.

Jika operator adjacency $A_N$ pada graf $G_N$ konvergen lokal pada operator 
adjacency $A_{T_3}$, maka resolventnya pun konvergen:
\[
\langle \delta_0, (A_N - zI)^{-1}\delta_0 \rangle
\to
\langle \delta_0, (A_{T_3} - zI)^{-1}\delta_0 \rangle.
\]

Karena $L_I$ adalah transformasi linear dari $A_N$, maka limitnya ditentukan.

Langkah penting:
\begin{enumerate}
    \item Membuktikan bahwa \emph{fluktuasi} lokal $A_N$ menghilang saat $N\to\infty$.
    \item Menggunakan teorema eksistensi limit spektral pohon regular.
    \item Menurunkan bentuk fungsi pita $\rho_\infty(\lambda)$ 
    dari resolvent pohon.
\end{enumerate}

%%%%%%%%%%%%%%%%%%%%%%%%%%%%%%%%%%%%%%%%%%%%%%%%%%%%%%%%%%%%%
\section{Metode 3: Heat Kernel dan Dimensi Spektral}
%%%%%%%%%%%%%%%%%%%%%%%%%%%%%%%%%%%%%%%%%%%%%%%%%%%%%%%%%%%%%

Definisikan heat kernel:
\begin{equation}
K_N(t) = \mathrm{Tr}\,e^{-t L_I}.
\end{equation}

Hubungan fundamental:
\begin{equation}
K_N(t)
=
\int e^{-t\lambda} \rho_N(\lambda)\, d\lambda.
\label{eq:heat-trace}
\end{equation}

Jika dapat dibuktikan:
\[
K_N(t) \to K_\infty(t),
\]
maka:
\[
\rho_N \to \rho_\infty.
\]

Pendekatan ini memerlukan:
\begin{enumerate}
    \item Batas atas heat kernel $K_N(t)$ dengan teknik Davies–Grigoryan.
    \item Ekspansi asimtotik jangka pendek:
    \[
    K_\infty(t) \sim t^{-2},
    \]
    yang memastikan dimensi spektral $d_s = 4$.
    \item Inversi transformasi Laplace untuk mendapatkan $\rho_\infty$.
\end{enumerate}

%%%%%%%%%%%%%%%%%%%%%%%%%%%%%%%%%%%%%%%%%%%%%%%%%%%%%%%%%%%%%
\section{Metode 4: Variasi Graf dan Stabilitas Pita}
%%%%%%%%%%%%%%%%%%%%%%%%%%%%%%%%%%%%%%%%%%%%%%%%%%%%%%%%%%%%%

Definisikan keluarga graf terdeformasi:
\[
G_N(\varepsilon)
\]
dengan matriks adjacency:
\[
A_N(\varepsilon) = A_N + \varepsilon B.
\]

Tujuan:
\[
\lambda_k(\varepsilon)
\text{ stabil terhadap } \varepsilon \to 0.
\]

Turunan pertama eigenvalue:
\begin{equation}
\frac{d\lambda_k}{d\varepsilon}
=
-\frac{2}{3}\langle \psi_k, B \psi_k\rangle.
\end{equation}

Untuk universalitas pita:
\[
\left|\frac{d\lambda_k}{d\varepsilon}\right|
\le C/N.
\]

Jika benar, fluktuasi tidak mengubah pita saat $N\to\infty$.

%%%%%%%%%%%%%%%%%%%%%%%%%%%%%%%%%%%%%%%%%%%%%%%%%%%%%%%%%%%%%
\section{Struktur Pembuktian yang Diperlukan}
%%%%%%%%%%%%%%%%%%%%%%%%%%%%%%%%%%%%%%%%%%%%%%%%%%%%%%%%%%%%%

Sebuah pembuktian lengkap USBC membutuhkan penyatuan hasil:

\begin{enumerate}
    \item Konvergensi resolvent (Metode 1).
    \item Konvergensi lokal graf (Metode 2).
    \item Konvergensi heat kernel (Metode 3).
    \item Stabilitas spektral terhadap deformasi (Metode 4).
\end{enumerate}

Jika keempat pendekatan memberi fungsi limit yang sama 
$\rho_\infty(\lambda)$, maka universalitas spektral terbukti.

%%%%%%%%%%%%%%%%%%%%%%%%%%%%%%%%%%%%%%%%%%%%%%%%%%%%%%%%%%%%%
\section{Kriteria Theorem Universalitas}
%%%%%%%%%%%%%%%%%%%%%%%%%%%%%%%%%%%%%%%%%%%%%%%%%%%%%%%%%%%%%

Teori Idris memiliki universalitas spektral jika dan hanya jika:

\begin{enumerate}
    \item Terdapat fungsi limit $\rho_\infty(\lambda)$ yang unik.
    \item $\rho_\infty(\lambda)$ berbentuk pita diskrit dan kontinu 
    sesuai prediksi.
    \item Ambang pita rendah memenuhi:
    \[
    \lambda_{\rm baryon} = \text{nilai kritis domain elektromagnetik}.
    \]
    \item Ambang pita tinggi memenuhi:
    \[
    \lambda_{\rm DM} = \text{nilai kritis stabilitas gravitasi}.
    \]
    \item Rasio degenerasi pita mengikuti struktur 1:3:9.
\end{enumerate}

%%%%%%%%%%%%%%%%%%%%%%%%%%%%%%%%%%%%%%%%%%%%%%%%%%%%%%%%%%%%%
\section{Kesimpulan Lampiran}
%%%%%%%%%%%%%%%%%%%%%%%%%%%%%%%%%%%%%%%%%%%%%%%%%%%%%%%%%%%%%

\begin{quote}
Pembuktian universalitas spektral dalam Teori Idris dapat dicapai melalui 
kombinasi analisis resolvent, konvergensi lokal graf, teknik heat kernel, 
dan stabilitas deformasi. 
Lampiran ini menyediakan kerangka metodologis rigor yang diperlukan untuk 
mengubah USBC dari conjecture menjadi theorem matematis penuh.
\end{quote}

\qed
